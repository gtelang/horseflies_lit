\documentclass[10pt,english, notitlepage, oneside]{report}
\usepackage[a4paper, 
            left=1.5cm, 
            right=1.5cm,
            bottom=1.5cm,
            ]{geometry}

%-------------------------------------------------------------------------------
\usepackage[left,pagewise]{lineno}
\linenumbers
\usepackage[framemethod=TikZ]{mdframed}
\usepackage{wrapfig}
\usepackage{amsthm}
\usepackage{amsfonts}

% ------------------------------------------------------------------------------
\usepackage[toc,page]{appendix}

%-------------------------------------------------------------------------------
%https://tex.stackexchange.com/a/278199/17858
% For aligning itemize environments to left
\usepackage{enumitem}
\mdfsetup{%
   middlelinecolor=red,
   middlelinewidth=0pt,
   roundcorner=10pt}

\mdfdefinestyle{MyFrame}{%
    linecolor=black,
    outerlinewidth=0.1pt,
    %roundcorner=20pt,
    innertopmargin=14pt,
    innerbottommargin=4pt,
    innerrightmargin=4pt,
    innerleftmargin=4pt,
        leftmargin = 4pt,
        rightmargin = 4pt,
    %backgroundcolor=gray!50!white}
     }
     
%-------------------------------------------------------------------------------
\definecolor{theocol}{RGB}{255, 222, 228}
\newtheorem{theo}{Theorem}
\newenvironment{ftheo}
  {\begin{mdframed}[style=MyFrame,nobreak=true, backgroundcolor=theocol]\begin{theo}}
  {\end{theo}\end{mdframed}}


\definecolor{lemcol}{RGB}{226, 255, 220}
\newtheorem{lem}{Lemma}
\newenvironment{flem}
  {\begin{mdframed}[style=MyFrame,nobreak=true, backgroundcolor=lemcol]\begin{lem}}
  {\end{lem}\end{mdframed}}

\definecolor{corcol}{RGB}{227, 227, 227}
\newtheorem{cor}{Corollary}
\newenvironment{fcor}
  {\begin{mdframed}[style=MyFrame,nobreak=true, backgroundcolor=corcol]\begin{cor}}
  {\end{cor}\end{mdframed}}

\definecolor{propcol}{RGB}{227, 227, 227}
\newtheorem{prop}{Proposition}
\newenvironment{fprop}
  {\begin{mdframed}[style=MyFrame,nobreak=true, backgroundcolor=propcol]\begin{prop}}
  {\end{prop}\end{mdframed}}

%-------------------------------------------------------------------------------
%https://tex.stackexchange.com/a/141662/17858
%reset numbering of theorems, lemmas and corollaries every chapter 
\usepackage{chngcntr}
\counterwithin*{theo}{chapter} 
\counterwithin*{lem}{chapter} 
\counterwithin*{cor}{chapter} 
\usepackage{enumitem}

%-------------------------------------------------------------------------------
% https://tex.stackexchange.com/q/2291/17858
% if you want to create a new list from scratch
\newlist{alphalist}{enumerate}{1}
% in that case, at least label must be specified using \setlist
\setlist[alphalist,1]{label=\textbf{\Alph*.}}

%--------------------------------------------------------------------------------
%\input{../OrgWeb/litproofs.tex}
\usepackage[english]{babel}
%\setlength{\voffset}{-0.75in}
\setlength{\headsep}{5pt}

%-------------------------------------------------------------------------------
\usepackage{blindtext}
\usepackage{lipsum}
\usepackage{textcomp}

%-------------------------------------------------------------------------------
\usepackage{asymptote}
\usepackage{parskip} 
\usepackage{marginnote}
\usepackage{tocloft}
\usepackage{graphicx}                             % Include images into pdf document 
\usepackage{tikz}                                 % Useful for the circled command. Don't remove!
\usepackage{multicol}                             % For lists in two or more columns
\usepackage[ruled, algochapter]{algorithm2e}                          % Typesetting pseudo-code
\newcommand\mycommfont[1]{\footnotesize\ttfamily\textcolor{blue}{#1}}
\SetCommentSty{mycommfont}
\SetKwComment{Comment}{$\triangleright$\ }{}

%-------------------------------------------------------------------------------
\usepackage[citecolor=red]{hyperref}  % Hyper-links in latex
\usepackage{needspace}                % So that sections/code-blocks don't straddle two pages 
\usepackage{mathtools}
\usepackage{subfig}
\usepackage{etoolbox}
\usepackage{color}
\usepackage{pifont}
\setlength{\parindent}{20pt}
% For italicizing quotes: https://tex.stackexchange.com/a/288556/17858

%-------------------------------------------------------------------------------
\usepackage[utf8]{inputenc}
\usepackage{csquotes}
\renewcommand{\mkbegdispquote}[2]{\itshape}

%-------------------------------------------------------------------------------
% Convenience commands
\newcommand*\circled[1]{\tikz[baseline=(char.base)]{\node[shape=circle,draw,inner sep=2pt] (char) {#1};}}
\providecommand{\myceil}[1]{\left \lceil #1 \right \rceil }	% Ceil function
\providecommand{\myfloor}[1]{\left \lfloor #1 \right \rfloor }	% Floor function\renewcommand{\labelitemi}{\tiny$\blacksquare$}	
\newcommand\given[1][]{\:#1\vert\:}                             % for drawing the conditional probability `|` sign neatly.
\newcommand\RR{\mathbb{R}}					% Set of reals numbers
\newcommand\CC{\mathbb{C}}					% Set of complex numbers
\newcommand\ZZ{\mathbb{Z}}					% Set of integers
\newcommand\NN{\mathbb{N}}					% Set of naturals
\newcommand\rarr{\rightarrow}					% Rightarrow
\newcommand\larr{\leftarrow}					% Leftarrow
\newcommand\defeq{\coloneqq}					% := symbol
\renewcommand\tilde{ \: \thicksim \: }				% Sane tildas
\newmdenv[topline=false, bottomline=false, skipabove=\topsep,skipbelow=\topsep]{siderules}
%\setlength{\droptitle}{-18em}					% Eliminate the default vertical space
\hypersetup{colorlinks=true,linkcolor=blue}


%-------------------------------------------------------------------------------
% Writing the text of a section immediately after section title
% Vedic and old british style where text immediately follows the 
% subsection number. https://tex.stackexchange.com/a/99177/17858
\usepackage{titlesec}
\titleformat{\section}[runin]{\normalfont\Large\bfseries}{\thesection}{0.5em}{}
\titleformat{\chapter}[hang] 
{\normalfont\huge\bfseries}{\chaptertitlename\ \thechapter:}{1em}{} 
%% Custom Literate Proof Commands:
%\newcommand{\newchunk}[2]{\section{#2}\label{sec:#1}\quad}
\newcommand{\newchunk}{\section{}\quad}
%\newcommand{\newchunk}[1]{\section{}\label{sec:#1}\quad}
\newcommand{\refchunk}[1]{\ref{sec:#1}}
\titlespacing{\section}{0pt}{*0}{*0}
\newcommand{\sect}[1]{\section{#1}\quad} 
%-------------------------------------------------------------------------------
% For centeriing the titile
% https://tex.stackexchange.com/a/290449/17858
\usepackage{titling}
\renewcommand\maketitlehooka{\null\mbox{}\vfill}
\renewcommand\maketitlehookd{\vfill\null}

%-------------------------------------------------------------------------------
% To express an idea in a crunchy way.
\newcommand{\crunchy}[1]{\lbrack{} \large \textit{#1} \normalsize \rbrack} 
%-------------------------------------------------------------------------------
\makeatletter
\def\@makechapterhead#1{%
  %%%%\vspace*{50\p@}% %%% removed!
  {\parindent \z@ \raggedright \normalfont
    \ifnum \c@secnumdepth >\m@ne
        \huge\bfseries \@chapapp\space \thechapter
        \par\nobreak
        \vskip 20\p@
    \fi
    \interlinepenalty\@M
    \Huge \bfseries #1\par\nobreak
    \vskip 40\p@
  }}
\def\@makeschapterhead#1{%
  %%%%%\vspace*{50\p@}% %%% removed!
  {\parindent \z@ \raggedright
    \normalfont
    \interlinepenalty\@M
    \Huge \bfseries  #1\par\nobreak
    \vskip 40\p@
  }}
\makeatother


%\renewcommand\cftchapafterpnum{\vskip10pt}
%\renewcommand\cftsecafterpnum{\vskip15pt}

%\usepackage{tocloft}
%\renewcommand{\cftchapleader}{\cftdotfill{\cftdotsep}}

% Fancy lettering at the beginning of a chapter
% Use the command as \lettrine{T}{he} the latter gets
% typeset in capitals.
% https://tex.stackexchange.com/a/256580/17858
\usepackage{lettrine}
\usepackage{GoudyIn}

\renewcommand{\LettrineFontHook}{\GoudyInfamily{}}
\LettrineTextFont{\itshape}
\setcounter{DefaultLines}{3}%

\usepackage{fancyhdr}
\usepackage{datetime}

\fancyhf{}
\fancyfoot[L]{\today\ \currenttime}
\pagestyle{fancy}
%------------------------------------------------------------------------------- 
%% Use this when you want table of contents in two columns, very useful! 
% \usepackage[toc]{multitoc}
% \renewcommand*{\multicolumntoc}{2}
% \usepackage[noautomatic,splitindex]{imakeidx}% For making indices: https://www.sharelatex.com/learn/Indices
% % Have chapter-wise indices and a full merged index.
% % You need to give a name for each chapter though. 
% % Whenver you insert anything, you should use the modified
% % Index command as described below. 
% %% To use this see: https://tex.stackexchange.com/a/434423/17858
% \makeindex[columns=3]
% \makeindex[title=Chapter one index, name=index1]% chapter 1 index
% \makeindex[title=Chapter two index, name=index2]% chapter 2 index
% \makeindex[title=Global Index, name=fullindex]    % main index
% % \Index{item}: add "item" to chapter and full index
% \newcommand\Index[1]{%
%    \index[index\arabic{chapter}]{#1}%  add to chapter index
%    \index[fullindex]{#1}%              add to full index
%  }

%    ------------------------------------------------------------------------------

\newcommand\lam{\lambda}
\newcommand\PP{\mathcal{P}}
\newcommand\Slam{\mathcal{S}_{\lambda}}
\newcommand\QQ{\mathcal{Q}}
\newcommand\Qlam{\QQ_{\lambda}}
\newcommand\gamSlam{{\gamma}_{\Slam}}
\newcommand\gamlam{{\gamma}_{\lambda}}
\newcommand\RRnonneg{\RR_{\geq 0}}  % Non-negative reals



% -------------------------------------------------------
% https://tex.stackexchange.com/a/30450/17858
% For styling the first pages of parts and chapters of a
% book. 
\usepackage{titlesec}

\titleclass{\part}{top} % make part like a chapter
\titleformat{\part}
[display]
{\centering\normalfont\Huge\bfseries}
{\titlerule[5pt]\vspace{3pt}\titlerule[2pt]\vspace{3pt}\MakeUppercase{\partname} \thepart}
{0pt}
{\titlerule[2pt]\vspace{1pc}\huge\MakeUppercase}
%
\titlespacing*{\part}{0pt}{0pt}{20pt}
%
\titleclass{\chapter}{straight} % make chapter like a section (no newpage)
\titleformat{\chapter}
[display]
{\centering\normalfont\Huge\bfseries}
{\titlerule[5pt]\vspace{3pt}\titlerule[2pt]\vspace{3pt}\MakeUppercase{\chaptertitlename} \thechapter}
{0pt}
{\titlerule[2pt]\vspace{6pt}\huge\MakeUppercase}

\titlespacing*{\chapter}{0pt}{0pt}{40pt}

% https://tex.stackexchange.com/a/458876/17858 rounded pink rectangles around an inline word
\newcommand{\sticker}[1]{\tikz[baseline=(X.base)]\node [draw=red,fill=pink!60,semithick,rectangle,inner sep=2pt, rounded corners=3pt] (X) { {\footnotesize \color{red} #1}};}
