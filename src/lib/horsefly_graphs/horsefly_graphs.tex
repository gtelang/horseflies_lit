\newcommand{\NWtarget}[2]{\hypertarget{#1}{#2}}
\newcommand{\NWlink}[2]{\hyperlink{#1}{#2}}
\newcommand{\NWtxtMacroDefBy}{Fragment defined by}
\newcommand{\NWtxtMacroRefIn}{Fragment referenced in}
\newcommand{\NWtxtMacroNoRef}{Fragment never referenced}
\newcommand{\NWtxtDefBy}{Defined by}
\newcommand{\NWtxtRefIn}{Referenced in}
\newcommand{\NWtxtNoRef}{Not referenced}
\newcommand{\NWtxtFileDefBy}{File defined by}
\newcommand{\NWtxtIdentsUsed}{Uses:}
\newcommand{\NWtxtIdentsNotUsed}{Never used}
\newcommand{\NWtxtIdentsDefed}{Defines:}
\newcommand{\NWsep}{${\diamond}$}
\newcommand{\NWnotglobal}{(not defined globally)}
\newcommand{\NWuseHyperlinks}{}
\documentclass[11.5pt]{report}
%\usepackage[a4paper, left=2cm, right=2cm]{geometry}
\usepackage[text={16cm,24cm}]{geometry}
\usepackage{marginnote}

% https://stackoverflow.com/a/27243065/505306
% Making caption font smaller on figures and tables.
\usepackage{caption}
\captionsetup{font=footnotesize}

% Forcing line-breaks in url's
% https://tex.stackexchange.com/questions/3033/forcing-linebreaks-in-url
\usepackage[hyphens]{url}

%-------------------------------------------------------------------------------
%\usepackage[left,pagewise]{lineno}
%\linenumbers
\usepackage[framemethod=TikZ]{mdframed}
\usepackage{wrapfig}
\usepackage{amsthm}
\usepackage{amsfonts}
\usepackage{minted}
% ------------------------------------------------------------------------------
\usepackage[toc,page]{appendix}

%-------------------------------------------------------------------------------
%https://tex.stackexchange.com/a/278199/17858
% For aligning itemize environments to left
\usepackage{enumitem}
\mdfsetup{%
   middlelinecolor=red,
   middlelinewidth=0pt,
   roundcorner=10pt}

\mdfdefinestyle{MyFrame}{%
    linecolor=black,
    outerlinewidth=0.1pt,
    roundcorner=0pt,
    innertopmargin=14pt,
    innerbottommargin=4pt,
    innerrightmargin=4pt,
    innerleftmargin=4pt,
        leftmargin = 4pt,
        rightmargin = 4pt,
    %backgroundcolor=gray!50!white}
     }
     
%-------------------------------------------------------------------------------

\definecolor{theocol}{RGB}{255, 222, 228}
\newtheorem{theo}{Theorem}
\newenvironment{ftheo}
  {\begin{mdframed}[style=MyFrame,nobreak=true, backgroundcolor=theocol]\begin{theo}}
  {\end{theo}\end{mdframed}}


\definecolor{lemcol}{RGB}{226, 255, 220}
\newtheorem{lem}{Lemma}
\newenvironment{flem}
  {\begin{mdframed}[style=MyFrame,nobreak=true, backgroundcolor=lemcol]\begin{lem}}
  {\end{lem}\end{mdframed}}

\definecolor{corcol}{RGB}{227, 227, 227}
\newtheorem{cor}{Corollary}
\newenvironment{fcor}
  {\begin{mdframed}[style=MyFrame,nobreak=true, backgroundcolor=corcol]\begin{cor}}
  {\end{cor}\end{mdframed}}

\definecolor{propcol}{RGB}{227, 227, 227}
\newtheorem{prop}{Proposition}
\newenvironment{fprop}
  {\begin{mdframed}[style=MyFrame,nobreak=true, backgroundcolor=propcol]\begin{prop}}
  {\end{prop}\end{mdframed}}

\definecolor{notecol}{RGB}{255, 152, 149  }
\newenvironment{note}
  {\begin{mdframed}[style=MyFrame,nobreak=true, backgroundcolor=notecol]}
  {\end{mdframed}}

\usepackage{enumitem}

%-------------------------------------------------------------------------------
% https://tex.stackexchange.com/q/2291/17858
% if you want to create a new list from scratch
\newlist{alphalist}{enumerate}{1}
% in that case, at least label must be specified using \setlist
\setlist[alphalist,1]{label=\textbf{\Alph*.}}

%--------------------------------------------------------------------------------
\usepackage[english]{babel}
%\setlength{\voffset}{-0.75in}
\setlength{\headsep}{5pt}

%-------------------------------------------------------------------------------
\usepackage{blindtext}
\usepackage{lipsum}
\usepackage{textcomp}

%-------------------------------------------------------------------------------
\usepackage{asymptote}
\usepackage{asypictureB}
\usepackage{filecontents}
\usepackage{parskip} 
\usepackage{tocloft}
\usepackage{graphicx}                             % Include images into pdf document 
\usepackage{tikz}                                 % Useful for the circled command. Don't remove!
\usepackage{multicol}                             % For lists in two or more columns

\usepackage[nokwfunc, ruled]{algorithm2e}
\newcommand\mycommfont[1]{\footnotesize\ttfamily\textcolor{blue}{#1}}
\SetCommentSty{mycommfont}
\SetKwComment{Comment}{$\triangleright$\ }{}
\usepackage[T1]{fontenc}
%-------------------------------------------------------------------------------
\usepackage{needspace}                % So that sections/code-blocks don't straddle two pages 
\usepackage{mathtools}
\usepackage{subfig}
\usepackage{etoolbox}
\usepackage{color}
\usepackage{pifont}
\setlength{\parindent}{20pt}
% For italicizing quotes: https://tex.stackexchange.com/a/288556/17858

%-------------------------------------------------------------------------------
\usepackage[utf8]{inputenc}
\usepackage{csquotes}
\renewcommand{\mkbegdispquote}[2]{\itshape}

%-------------------------------------------------------------------------------
% Convenience commands
\newcommand*\circled[1]{\tikz[baseline=(char.base)]{\node[shape=circle,draw,inner sep=2pt] (char) {#1};}}
\providecommand{\myceil}[1]{\left \lceil #1 \right \rceil }	% Ceil function
\providecommand{\myfloor}[1]{\left \lfloor #1 \right \rfloor }	% Floor function\renewcommand{\labelitemi}{\tiny$\blacksquare$}	
\newcommand\given[1][]{\:#1\vert\:}                             % for drawing the conditional probability `|` sign neatly.
\newcommand\RR{\mathbb{R}}					% Set of reals numbers
\newcommand\CC{\mathbb{C}}					% Set of complex numbers
\newcommand\ZZ{\mathbb{Z}}					% Set of integers
\newcommand\NN{\mathbb{N}}					% Set of naturals
\newcommand\rarr{\rightarrow}					% Rightarrow
\newcommand\larr{\leftarrow}					% Leftarrow
\newcommand\defeq{\coloneqq}					% := symbol
\renewcommand\tilde{ \: \thicksim \: }				% Sane tildas
\newmdenv[topline=false, bottomline=false, skipabove=\topsep,skipbelow=\topsep]{siderules}
%\setlength{\droptitle}{-18em}					% Eliminate the default vertical space
%\hypersetup{colorlinks=true,linkcolor=blue}


%-------------------------------------------------------------------------------
% Writing the text of a section immediately after section title
% Vedic and old british style where text immediately follows the 
% subsection number. https://tex.stackexchange.com/a/99177/17858
%\usepackage{titlesec}
%\titlespacing{\section}{0pt}{*0}{*0}
%\titlespacing{\subsection}{0pt}{*0}{*0}
%\titlespacing{\subsubsection}{0pt}{*0}{*0}
\usepackage[compact]{titlesec}
\titlespacing{\section}{0pt}{2ex}{1ex}
\titlespacing{\subsection}{0pt}{1ex}{0ex}
\titlespacing{\subsubsection}{0pt}{0.5ex}{0ex}
%\setlength{\parskip}{0.1cm} % Paragraph separation
\setlength{\parindent}{2em}

\titleformat*{\section}{\Huge\bfseries}
\titleformat{\subsection}[runin]{\normalfont\Large\bfseries}{\thesubsection}{0.5em}{}
\titleformat*{\subsubsection}{\Large\bfseries}
%{\normalfont\huge\bfseries}{\sectiontitlename\ \thesection:}{1em}{} 
%% Custom Literate Proof Commands:
%\newcommand{\newchunk}[2]{\section{#2}\label{sec:#1}\quad}
\newcommand{\newchunk}{\subsection{}\quad}
\newcommand{\refchunk}[1]{\ref{sec:#1}}
%\newcommand{\sect}[1]{\section{#1}\quad} 
%-------------------------------------------------------------------------------
% For centeriing the titile
% https://tex.stackexchange.com/a/290449/17858
\usepackage{titling}
\renewcommand\maketitlehooka{\null\mbox{}\vfill}
\renewcommand\maketitlehookd{\vfill\null}

%-------------------------------------------------------------------------------
% To express an idea in a crunchy way.
\newcommand{\crunchy}[1]{\lbrack{} \large \textit{#1} \normalsize \rbrack} 
%-------------------------------------------------------------------------------
\makeatletter
\def\@makechapterhead#1{%
  %%%%\vspace*{50\p@}% %%% removed!
  {\parindent \z@ \raggedright \normalfont
    \ifnum \c@secnumdepth >\m@ne
        \huge \bfseries \@chapapp\space \thechapter
        \par\nobreak
        \vskip 20\p@
    \fi
    \interlinepenalty\@M
    \Huge \bfseries #1\par\nobreak
    \vskip 40\p@
  }}
\def\@makeschapterhead#1{%
  %%%%%\vspace*{50\p@}% %%% removed!
  {\parindent \z@ \raggedright
    \normalfont
    \interlinepenalty\@M
    \Huge \bfseries  #1\par\nobreak
    \vskip 40\p@
  }}
\makeatother


\usepackage{fancyhdr}
\usepackage{datetime}

\fancyhf{}
\fancyfoot[L]{\today\ \currenttime}
\pagestyle{fancy}

% https://tex.stackexchange.com/a/458876/17858 rounded
% pink rectangles around an inline word
\newcommand{\sticker}[1]{\tikz[baseline=(X.base)]\node [draw=red,fill=pink!60,semithick,rectangle,inner sep=2pt, rounded corners=3pt] (X) { {\footnotesize \color{red} #1}};}


\newif\ifshowcode
\showcodetrue
\usepackage{latexsym}
\usepackage{listings}
\usepackage{color}
\definecolor{linkcolor}{rgb}{0, 0, 0.7}

\usepackage[%
backref,%
raiselinks,%
pdfhighlight=/O,%
pagebackref,%
hyperfigures,%
breaklinks,%
colorlinks,%
pdfpagemode=None,%
pdfstartview=FitBH,%
linkcolor={linkcolor},%
anchorcolor={linkcolor},%
citecolor={linkcolor},%
filecolor={linkcolor},%
menucolor={linkcolor},%
pagecolor={linkcolor},%
urlcolor={linkcolor}%
]{hyperref}


% % %-------------------------------------------------------------------------------
% % %https://tex.stackexchange.com/a/141662/17858
% % %reset numbering of theorems, lemmas and corollaries every chapter 
% \usepackage{chngcntr}
% \counterwithin*{theo}{chapter} 
% \counterwithin*{lem}{chapter} 
% \counterwithin*{cor}{chapter} 


%% For making lualatex work with asymptote
%% ripped from https://tex.stackexchange.com/a/376412/17858
\usepackage{ifluatex}
\ifluatex
\makeatletter
\def\asy@input@graphic{%
  \ifASYinline
    \IfFileExists{"\AsyFile.tex"}{%
      \catcode`:=12\relax
      \@@input"\AsyFile.tex"\relax
    }{%
      \PackageWarning{asymptote}{file `\AsyFile.tex' not found}%
    }%
  \else
    \IfFileExists{"\AsyFile.\AsyExtension"}{%
      \ifASYattach
        \ifASYPDF
          \IfFileExists{"\AsyFile+0.pdf"}{%
            \setbox\ASYbox=\hbox{\includegraphics[hiresbb]{\AsyFile+0.pdf}}%
          }{%
            \setbox\ASYbox=\hbox{\includegraphics[hiresbb]{\AsyFile.pdf}}%
          }%
        \else
          \setbox\ASYbox=\hbox{\includegraphics[hiresbb]{\AsyFile.eps}}%
        \fi
        \textattachfile{\AsyFile.\AsyExtension}{\phantom{\copy\ASYbox}}%
        \vskip-\ht\ASYbox
        \indent
        \box\ASYbox
      \else
        \ifASYPDF
          \includegraphics[hiresbb]{\AsyFile.pdf}%
        \else
          \includegraphics[hiresbb]{\AsyFile.eps}%
        \fi
      \fi
    }{%
      \IfFileExists{"\AsyFile.tex"}{%
        \catcode`:=12
        \@@input"\AsyFile.tex"\relax
      }{%
        \PackageWarning{asymptote}{%
          file `\AsyFile.\AsyExtension' not found%
        }%
      }%
    }%
  \fi
}
\makeatother
\fi

%%%%%%%%%%%%%%%%%%%%%%%%%%%%%%%%%%%%%%%%%%%%%%%%%%%%%%%%%%%%%%%%
%% for setting program fonts
%% As recommended by http://nepsweb.co.uk/docs/progfonts.pdf
%% He recommends bera mono. 
%%\usepackage[scaled]{beramono}
%%\usepackage[T1]{fontenc}

%% I liked txtt also. 
\renewcommand{\ttdefault}{txtt}
                                                                                                                      
%%% Super useful for marking todo notes, ripped from here:                                                            
%%% https://tex.stackexchange.com/a/178806/17858                                                                      
\usepackage{xargs}                      % Use more than one optional parameter in a new commands                      
\usepackage[colorinlistoftodos,prependcaption,textsize=tiny]{todonotes}
\newcommandx{\UNSURE}[2][1=]{\todo[linecolor=blue,backgroundcolor=blue!25,bordercolor=blue,#1]{#2}}
\newcommandx{\TODO}[2][1=]{\todo[linecolor=red,backgroundcolor=red!25,bordercolor=red,#1]{#2}}
                                                                                                                      
\usepackage{kantlipsum}
\usepackage{fancyvrb}
\usepackage{setspace}
\newenvironment{CVerbatim}
 {\singlespacing\center\BVerbatim}  
 {\endBVerbatim\endcenter}
                                                                                                                      
\usepackage{tocloft}
\renewcommand{\cftpartfont}{\LARGE\itshape} % Part title in Huge Italic font                                          
\usepackage{hyperref}
\usepackage{etoolbox}
% Better formatting of backticks in                                                                                   
% verbatim environment.                                                                                               
\usepackage{upquote} 
                                                                                                                      
% page numbering at top right                                                                                         
\usepackage{fancyhdr}
\pagestyle{fancy} 
\fancyhf{}                                                                                                            
\fancyhead[R]{\thepage}                                                                                               
  
\title{Horsefly Problems on Graphs}

\begin{document}
\maketitle
\tableofcontents

\chapter{Introduction}

This set of literate programs implement heuristics to solve Horsefly type problems on graphs. The generic problem for this setting 
is: given a set of nodes, with some distinguished nodes marked as delivery points and one node marked as the starting point 
of a truck and drone we want to minimize the makespan of the delivery process in which the drone departs from the truck 
with a package, flies to a site along a sequence of edges, drops of the package and returns to the truck (which itself might be moving 
along the graph edges) to pick up the next package for delivery. For the sake of simplicity, we only allow the drone itself to do the delivery 
and the meeting points between the truck and drone to happen at the nodes of the graph. Some edges in the graph will be traversable by 
only the truck or only the drone and some edges will be traversable by both. 

The challenge in this problem is to set up a coordinated schedule of the truck and the drone so that the total delivery time is minimized. 
The truck is assumed to always travel with unit speed and the drone with integer speed $\varphi \in \{2,3,\ldots\}$. 

While almost all the algorithms developed here are intended to be graph agnostic, the input graphs for such problems will typically come 
about from the discretization of geometric domains where there might be obstacles, some for the drone, some for the truck, some for maybe both. 
To test the algorithms interactively, I will typically use rectangular obstacles, along with an appropriate background mesh of points.



Several variants on this basic theme are possible in which a truck might have been equipped with more than one drone or there might be 
constraints at the sites or pick-up points (such as service or package-handoff times). Lots of special cases abound too: what if the underlying 
graph is or outerplanar planar, what if the set of truck and/or drone edges form a tree? What if the truck is restricted to travelling only 
along a path-graph? 

Heuristics for such variants and their experimental-performance analyses 
will also be discussed later. Such problems are typically extremely hard, so all heuristics are meant to be approximations to the optimal 
routing scheme for each instance. Algorithms developed will typically use a combination of greedy strategies, dynamic programming, branch-and-bound
and local search. Several interesting problems also arise in being able to efficiently implement a particular strategy and so answers to these questions
will often be developed in tandem with the implementation of different strategies for Horsefly problems. 


\chapter{User Input}

This chapter implements (boring  but important) UI code for the user to mouse in rectangular obstacles, service points, and initial position of the truck and drone onto 
the Matplotlib canvas and generate an appropriate geometric graph for the truck and drone to travel along. Typically these geometric graphs 
will be the visibility graph, delaunay triangulation or some other graph on such points which can be set at run-time through a command-line menu. 
The resulting algorithms, however, will typically be graph-agnostic unless explicitly indicated in the doc-string of the implemented algorithm. 

One can also enter in the set of truck and drone obstacles along with any discretization parameters via a YAML file. This is particularly useful 
for storing nice instances and answers discovered wihle experimenting on a canvas. 

Here is a high-level overview of the \verb|horsefly_graphs.py| module. 

%{python-mode}%
\begin{flushleft} \small\label{scrap1}\raggedright\small
\NWtarget{nuweb3}{} \verb@"horsefly_graphs.py"@\nobreak\ {\footnotesize {3}}$\equiv$
\vspace{-1ex}
\begin{list}{}{} \item
\mbox{}\verb@    @\\
\mbox{}\verb@import matplotlib.pyplot as plt@\\
\mbox{}\verb@import matplotlib as mpl@\\
\mbox{}\verb@from matplotlib import rc@\\
\mbox{}\verb@import numpy as np@\\
\mbox{}\verb@import scipy as sp@\\
\mbox{}\verb@import sys, os, time@\\
\mbox{}\verb@import itertools@\\
\mbox{}\verb@from colorama import Fore, Style@\\
\mbox{}\verb@import logging@\\
\mbox{}\verb@import utils_graphics, utils_algo@\\
\mbox{}\verb@import networkx as nx@\\
\mbox{}\verb@from CGAL.CGAL_Kernel import Point_2, Segment_2, Iso_rectangle_2@\\
\mbox{}\verb@from CGAL.CGAL_Kernel import do_intersect, intersection@\\
\mbox{}\verb@@\\
\mbox{}\verb@@\hbox{$\langle\,${\itshape Local data-structures}\nobreak\ {\footnotesize \NWlink{nuweb4}{4}, \ldots\ }$\,\rangle$}\verb@@\\
\mbox{}\verb@@\hbox{$\langle\,${\itshape Some basic canvas functions}\nobreak\ {\footnotesize \NWlink{nuweb6a}{6a}}$\,\rangle$}\verb@@\\
\mbox{}\verb@@\hbox{$\langle\,${\itshape Implementation of the key-press handler}\nobreak\ {\footnotesize \NWlink{nuweb7a}{7a}}$\,\rangle$}\verb@@\\
\mbox{}\verb@@\hbox{$\langle\,${\itshape Implementation of \verb|wrapperEnterPoints|}\nobreak\ {\footnotesize \NWlink{nuweb7b}{7b}}$\,\rangle$}\verb@@\\
\mbox{}\verb@@\hbox{$\langle\,${\itshape Implementation of \verb|wrapperHoverObstacle|}\nobreak\ {\footnotesize \NWlink{nuweb8}{8}}$\,\rangle$}\verb@@\\
\mbox{}\verb@@\hbox{$\langle\,${\itshape Implementation of \verb|wrapperPlaceObstacle|}\nobreak\ {\footnotesize \NWlink{nuweb9}{9}}$\,\rangle$}\verb@   @\\
\mbox{}\verb@@\hbox{$\langle\,${\itshape Implementation of \verb|wrapperResizeHoveringObstacle|}\nobreak\ {\footnotesize \NWlink{nuweb10}{10}}$\,\rangle$}\verb@   @\\
\mbox{}\verb@@\\
\mbox{}\verb@if __name__ == "__main__":@\\
\mbox{}\verb@    @\hbox{$\langle\,${\itshape Body of main function}\nobreak\ {\footnotesize \NWlink{nuweb6b}{6b}}$\,\rangle$}\verb@@\\
\mbox{}\verb@@{\NWsep}
\end{list}
\vspace{-1.5ex}
\footnotesize
\begin{list}{}{\setlength{\itemsep}{-\parsep}\setlength{\itemindent}{-\leftmargin}}

\item{}
\end{list}
\vspace{4ex}
\end{flushleft}
%{/python-mode}%


\section{Local Data Structures}

The data-structure used by the canvas for manipulating the input is \verb|HorseflyInputGraph|. It performs simple house-keeping duties such as keeping track of the 
sites, the initial position of the truck, various obstacles etc. The main method for this class is \verb|getTour| which generates the actual graph according to 
the graph-policy specified and runs the appropriate algorithm on the generated graph. The flag \verb|obstacle\_input\_mode\_p| is toggled to \verb|True| or 
\verb|False| according as obstacles are being entered onto the canvas. The input is entered in two modes (the default site insertion mode and the optional 
obstacle insertion mode, which is toggled by pressing the ``o'' or ``O'' key. )

The definition of the method \verb|makeHorseflyInputGraph| will be given later. This will help exposition since details of the function depend on the 
\verb|Obstacle| class which will be introduced in the next code-chunk.  


%{python-mode}%
\begin{flushleft} \small\label{scrap2}\raggedright\small
\NWtarget{nuweb4}{} $\langle\,${\itshape Local data-structures}\nobreak\ {\footnotesize {4}}$\,\rangle\equiv$
\vspace{-1ex}
\begin{list}{}{} \item
\mbox{}\verb@@\\
\mbox{}\verb@ @\\
\mbox{}\verb@class HorseflyInputGraph:@\\
\mbox{}\verb@      def __init__(self, sites=[], inithorseposn=[]):@\\
\mbox{}\verb@           self.sites                = sites@\\
\mbox{}\verb@           self.inithorseposn        = inithorseposn@\\
\mbox{}\verb@           self.obstacle_list        = []@\\
\mbox{}\verb@           self.horsefly_input_graph = nx.Graph()@\\
\mbox{}\verb@    @\\
\mbox{}\verb@           # Tracking variables used only during interactive input. @\\
\mbox{}\verb@           # To be frank these tracking variables should be placed in a separate class@\\
\mbox{}\verb@           # they seem out of place here. Else have a dedicated class stored as a @\\
\mbox{}\verb@           # variable/dictionary containinig all these icky state variables used @\\
\mbox{}\verb@           # during input manipulation. @\\
\mbox{}\verb@           self.default_obstacle_width   = 0.1 @\\
\mbox{}\verb@           self.default_obstacle_height  = 0.1 @\\
\mbox{}\verb@@\\
\mbox{}\verb@           self.hovering_obstacle_width  = 0.1 # This will be modified during the move@\\
\mbox{}\verb@           self.hovering_obstacle_height = 0.1 # This will be modified during the move@\\
\mbox{}\verb@@\\
\mbox{}\verb@           self.horse_obstacle_input_mode_p = False@\\
\mbox{}\verb@           self.fly_obstacle_input_mode_p = False@\\
\mbox{}\verb@           self.common_obstacle_input_mode_p = False@\\
\mbox{}\verb@@\\
\mbox{}\verb@@\\
\mbox{}\verb@      def clearAllStates (self):@\\
\mbox{}\verb@          self.sites           = []@\\
\mbox{}\verb@          self.inithorseposn   = None@\\
\mbox{}\verb@          self.obstacle_list = [] @\\
\mbox{}\verb@   @\\
\mbox{}\verb@          self.horse_obstacle_input_mode_p  = False@\\
\mbox{}\verb@          self.fly_obstacle_input_mode_p    = False@\\
\mbox{}\verb@          self.common_obstacle_input_mode_p = False@\\
\mbox{}\verb@    @\\
\mbox{}\verb@          self.horsefly_input_graph = nx.Graph()@\\
\mbox{}\verb@      @\\
\mbox{}\verb@      @\hbox{$\langle\,${\itshape Definition of method \verb|makeHorseflyInputGraph|}\nobreak\ {\footnotesize \NWlink{nuweb11a}{11a}}$\,\rangle$}\verb@@\\
\mbox{}\verb@      @\hbox{$\langle\,${\itshape Definition of method \verb|renderHorseflyInputGraph|}\nobreak\ {\footnotesize \NWlink{nuweb15b}{15b}}$\,\rangle$}\verb@@\\
\mbox{}\verb@@\\
\mbox{}\verb@      def clearGraph(self,fig,ax):@\\
\mbox{}\verb@          self.horsefly_input_graph = nx.Graph() # Input Graph is set to the empty null graph@\\
\mbox{}\verb@          # TODO, clear the canvas completely and then redraw the current sites and obstacles stored@\\
\mbox{}\verb@          # so that the drawn graph is cleared from the canvas, and the input is restored to its @\\
\mbox{}\verb@          # original prsitine state, so that other algorithms can be tried out. @\\
\mbox{}\verb@          pass@\\
\mbox{}\verb@@\\
\mbox{}\verb@      def getTour(self, algo, phi):@\\
\mbox{}\verb@             return algo(self.horsefly_input_graph, phi)@\\
\mbox{}\verb@@\\
\mbox{}\verb@@\\
\mbox{}\verb@@{\NWsep}
\end{list}
\vspace{-1.5ex}
\footnotesize
\begin{list}{}{\setlength{\itemsep}{-\parsep}\setlength{\itemindent}{-\leftmargin}}
\item \NWtxtMacroDefBy\ \NWlink{nuweb4}{4}\NWlink{nuweb5}{, 5}.
\item \NWtxtMacroRefIn\ \NWlink{nuweb3}{3}.

\item{}
\end{list}
\vspace{4ex}
\end{flushleft}
%{/python-mode}%

The UI code allows only rectangular obstacles to be inserted for convenience of implementation. As has already been mentioned, once the underlying domain of obsctales and surrounding
space is discretized the algorithms are graph agnostic. Code for inserting obstacles of more complicated shapes will be added later. 

%{python-mode}%
\begin{flushleft} \small\label{scrap3}\raggedright\small
\NWtarget{nuweb5}{} $\langle\,${\itshape Local data-structures}\nobreak\ {\footnotesize {5}}$\,\rangle\equiv$
\vspace{-1ex}
\begin{list}{}{} \item
\mbox{}\verb@@\\
\mbox{}\verb@class Obstacle:@\\
\mbox{}\verb@    def __init__(self, llcorner, width, height, figure=None, @\\
\mbox{}\verb@                 axes_object=None, obstype=None, diskcolor = 'crimson'):@\\
\mbox{}\verb@        self.llcorner = llcorner@\\
\mbox{}\verb@        self.width    = width@\\
\mbox{}\verb@        self.height   = height@\\
\mbox{}\verb@        self.fig      = figure@\\
\mbox{}\verb@        self.ax       = axes_object@\\
\mbox{}\verb@        self.obstype  = obstype@\\
\mbox{}\verb@@\\
\mbox{}\verb@        # Only inserted disks will end up having this attribute@\\
\mbox{}\verb@        if self.fig != None:@\\
\mbox{}\verb@             self.canvas_patch =  mpl.patches.Rectangle( self.llcorner, self.width, @\\
\mbox{}\verb@                                                         self.height  , facecolor = diskcolor, @\\
\mbox{}\verb@                                                         alpha = 0.7 )@\\
\mbox{}\verb@@\\
\mbox{}\verb@    def mplPatch(self, diskcolor= 'crimson' ):@\\
\mbox{}\verb@        return  self.canvas_patch@\\
\mbox{}\verb@@\\
\mbox{}\verb@    def getVertices(self):@\\
\mbox{}\verb@         [x,y] = self.llcorner@\\
\mbox{}\verb@         p_ll = [x            , y       ]@\\
\mbox{}\verb@         p_lr = [x+self.width , y       ]@\\
\mbox{}\verb@         p_ur = [x+self.width , y+self.height]@\\
\mbox{}\verb@         p_ul = [x            , y+self.height]@\\
\mbox{}\verb@         return [p_ll,p_lr,p_ur,p_ul]@\\
\mbox{}\verb@@\\
\mbox{}\verb@    def intersectionWithSegment(self,p,q):@\\
\mbox{}\verb@    @\\
\mbox{}\verb@        [llv, _, urv, _] = self.getVertices()@\\
\mbox{}\verb@        llv = Point_2(llv[0],llv[1])@\\
\mbox{}\verb@        urv = Point_2(urv[0],urv[1])@\\
\mbox{}\verb@        rect = Iso_rectangle_2(llv,urv)@\\
\mbox{}\verb@@\\
\mbox{}\verb@        p = Point_2(p[0],p[1])@\\
\mbox{}\verb@        q = Point_2(q[0],q[1])@\\
\mbox{}\verb@        seg = Segment_2(p,q)@\\
\mbox{}\verb@@\\
\mbox{}\verb@        interscn_object = intersection(rect,seg)@\\
\mbox{}\verb@        interscn_p     = do_intersect(rect,seg)@\\
\mbox{}\verb@@\\
\mbox{}\verb@        return (interscn_p, interscn_object)@\\
\mbox{}\verb@@{\NWsep}
\end{list}
\vspace{-1.5ex}
\footnotesize
\begin{list}{}{\setlength{\itemsep}{-\parsep}\setlength{\itemindent}{-\leftmargin}}
\item \NWtxtMacroDefBy\ \NWlink{nuweb4}{4}\NWlink{nuweb5}{, 5}.
\item \NWtxtMacroRefIn\ \NWlink{nuweb3}{3}.

\item{}
\end{list}
\vspace{4ex}
\end{flushleft}
%{/python-mode}%

\section{Some basic canvas functions}

A few basic functions are needed to manipulate the matplotlib canvas, such as clearing the canvas of patches, setting aspect ratio etc. We implement them here. 

%{python-mode}%
\begin{flushleft} \small
\begin{minipage}{\linewidth}\label{scrap4}\raggedright\small
\NWtarget{nuweb6a}{} $\langle\,${\itshape Some basic canvas functions}\nobreak\ {\footnotesize {6a}}$\,\rangle\equiv$
\vspace{-1ex}
\begin{list}{}{} \item
\mbox{}\verb@@\\
\mbox{}\verb@def applyAxCorrection(ax):@\\
\mbox{}\verb@      ax.set_xlim([utils_graphics.xlim[0], utils_graphics.xlim[1]])@\\
\mbox{}\verb@      ax.set_ylim([utils_graphics.ylim[0], utils_graphics.ylim[1]])@\\
\mbox{}\verb@      ax.set_aspect(1.0)@\\
\mbox{}\verb@@\\
\mbox{}\verb@def clearPatches(ax):@\\
\mbox{}\verb@    # Get indices cooresponding to the polygon patches@\\
\mbox{}\verb@    for index , patch in zip(range(len(ax.patches)), ax.patches):@\\
\mbox{}\verb@        if isinstance(patch, mpl.patches.Polygon) == True:@\\
\mbox{}\verb@            patch.remove()@\\
\mbox{}\verb@    ax.lines[:]=[]@\\
\mbox{}\verb@    applyAxCorrection(ax)@\\
\mbox{}\verb@@\\
\mbox{}\verb@def clearAxPolygonPatches(ax):@\\
\mbox{}\verb@@\\
\mbox{}\verb@    # Get indices cooresponding to the polygon patches@\\
\mbox{}\verb@    for index , patch in zip(range(len(ax.patches)), ax.patches):@\\
\mbox{}\verb@        if isinstance(patch, mpl.patches.Polygon) == True:@\\
\mbox{}\verb@            patch.remove()@\\
\mbox{}\verb@    ax.lines[:]=[]@\\
\mbox{}\verb@    applyAxCorrection(ax)@\\
\mbox{}\verb@@\\
\mbox{}\verb@@{\NWsep}
\end{list}
\vspace{-1.5ex}
\footnotesize
\begin{list}{}{\setlength{\itemsep}{-\parsep}\setlength{\itemindent}{-\leftmargin}}
\item \NWtxtMacroRefIn\ \NWlink{nuweb3}{3}.

\item{}
\end{list}
\end{minipage}\vspace{4ex}
\end{flushleft}
%{/python-mode}%


The \verb|main| function just sets up the basic canvas and the container to hold the user-input data. Once the canvas is started, it ``listens in'' on the mouse-press and key-press events 
to enter various elements such as the initial position of the truck and drone, the sites of delivery, the input obstacles. 

%{python-mode}%
\begin{flushleft} \small\label{scrap5}\raggedright\small
\NWtarget{nuweb6b}{} $\langle\,${\itshape Body of main function}\nobreak\ {\footnotesize {6b}}$\,\rangle\equiv$
\vspace{-1ex}
\begin{list}{}{} \item
\mbox{}\verb@@\\
\mbox{}\verb@fig, ax =  plt.subplots()@\\
\mbox{}\verb@run = HorseflyInputGraph()@\\
\mbox{}\verb@        @\\
\mbox{}\verb@ax.set_xlim([utils_graphics.xlim[0], utils_graphics.xlim[1]])@\\
\mbox{}\verb@ax.set_ylim([utils_graphics.ylim[0], utils_graphics.ylim[1]])@\\
\mbox{}\verb@ax.set_aspect(1.0)@\\
\mbox{}\verb@ax.set_xticks([])@\\
\mbox{}\verb@ax.set_yticks([])@\\
\mbox{}\verb@          @\\
\mbox{}\verb@mouseClick = wrapperEnterPoints (fig,ax, run)@\\
\mbox{}\verb@fig.canvas.mpl_connect('button_press_event' , mouseClick )@\\
\mbox{}\verb@         @\\
\mbox{}\verb@keyPress   = wrapperkeyPressHandler(fig,ax, run)@\\
\mbox{}\verb@fig.canvas.mpl_connect('key_press_event', keyPress )@\\
\mbox{}\verb@    @\\
\mbox{}\verb@hoverObstacle  = wrapperHoverObstacle(fig,ax, run)@\\
\mbox{}\verb@fig.canvas.mpl_connect('motion_notify_event', hoverObstacle )@\\
\mbox{}\verb@    @\\
\mbox{}\verb@placeObstacle  = wrapperPlaceObstacle(fig,ax, run)@\\
\mbox{}\verb@fig.canvas.mpl_connect('button_press_event' , placeObstacle )@\\
\mbox{}\verb@@\\
\mbox{}\verb@resizeHoveringObstacle  = wrapperResizeHoveringObstacle(fig,ax, run)@\\
\mbox{}\verb@fig.canvas.mpl_connect('key_press_event', resizeHoveringObstacle)@\\
\mbox{}\verb@@\\
\mbox{}\verb@plt.show()@\\
\mbox{}\verb@@{\NWsep}
\end{list}
\vspace{-1.5ex}
\footnotesize
\begin{list}{}{\setlength{\itemsep}{-\parsep}\setlength{\itemindent}{-\leftmargin}}
\item \NWtxtMacroRefIn\ \NWlink{nuweb3}{3}.

\item{}
\end{list}
\vspace{4ex}
\end{flushleft}
%{/python-mode}%

This function handles the manages events corresponding to key-presses on the computer key-board. For instance, 
pressing \verb|Ctrl+| $\rightarrow$ increases (resp. decreases) the X-dimension of the rectangle. Similarly for 
the Y-dimension and the keys \verb|Ctrl+| $\uparrow$ and \verb|Ctrl+| $\downarrow$. Behavior of other-keys is 
self-explantory as documented in the code-chunks below.  

%{python-mode}%
\begin{flushleft} \small\label{scrap6}\raggedright\small
\NWtarget{nuweb7a}{} $\langle\,${\itshape Implementation of the key-press handler}\nobreak\ {\footnotesize {7a}}$\,\rangle\equiv$
\vspace{-1ex}
\begin{list}{}{} \item
\mbox{}\verb@@\\
\mbox{}\verb@def wrapperkeyPressHandler(fig,ax, run): @\\
\mbox{}\verb@      def _keyPressHandler(event):@\\
\mbox{}\verb@                      @\\
\mbox{}\verb@             if event.key in ['c', 'C']: @\\
\mbox{}\verb@                   # Clear canvas and states of all objects@\\
\mbox{}\verb@                    run.clearAllStates()@\\
\mbox{}\verb@                    ax.cla()@\\
\mbox{}\verb@                                  @\\
\mbox{}\verb@                    utils_graphics.applyAxCorrection(ax)@\\
\mbox{}\verb@                    ax.set_xticks([])@\\
\mbox{}\verb@                    ax.set_yticks([])@\\
\mbox{}\verb@                                     @\\
\mbox{}\verb@                    fig.texts = []@\\
\mbox{}\verb@                    fig.canvas.draw()@\\
\mbox{}\verb@             @\\
\mbox{}\verb@             elif event.key in ['h' , 'H']: # `h` for horse@\\
\mbox{}\verb@                  run.horse_obstacle_input_mode_p    = not (run.horse_obstacle_input_mode_p)@\\
\mbox{}\verb@                  run.fly_obstacle_input_mode_p      = False@\\
\mbox{}\verb@                  run.common_obstacle_input_mode_p   = False@\\
\mbox{}\verb@@\\
\mbox{}\verb@    @\\
\mbox{}\verb@             elif event.key in ['f' , 'F']: # `f` for fly@\\
\mbox{}\verb@                  run.horse_obstacle_input_mode_p   = False@\\
\mbox{}\verb@                  run.fly_obstacle_input_mode_p     = not (run.fly_obstacle_input_mode_p)@\\
\mbox{}\verb@                  run.common_obstacle_input_mode_p  = False@\\
\mbox{}\verb@@\\
\mbox{}\verb@@\\
\mbox{}\verb@             elif event.key in ['g' , 'G']: # 'g' lies between the f and h keys on the keybard signifying intersection@\\
\mbox{}\verb@                  run.horse_obstacle_input_mode_p   = False@\\
\mbox{}\verb@                  run.fly_obstacle_input_mode_p     = False@\\
\mbox{}\verb@                  run.common_obstacle_input_mode_p  = not (run.common_obstacle_input_mode_p)@\\
\mbox{}\verb@    @\\
\mbox{}\verb@             elif event.key in ['d','D']: # `d` for discretize domain, using the obstacles we sprinkle @\\
\mbox{}\verb@                                          # some points and discretize everything@\\
\mbox{}\verb@                  background_grid_pts = [[np.random.rand(), np.random.rand()] for i in range(200)]@\\
\mbox{}\verb@                  run.makeHorseflyInputGraph(fig, ax, background_grid_pts, k=5)@\\
\mbox{}\verb@@\\
\mbox{}\verb@      return _keyPressHandler@\\
\mbox{}\verb@@{\NWsep}
\end{list}
\vspace{-1.5ex}
\footnotesize
\begin{list}{}{\setlength{\itemsep}{-\parsep}\setlength{\itemindent}{-\leftmargin}}
\item \NWtxtMacroRefIn\ \NWlink{nuweb3}{3}.

\item{}
\end{list}
\vspace{4ex}
\end{flushleft}
%{/python-mode}%

This function just implements the behavior of the canvas under mouse-clicks. The left mouse button double-clicked inserts a site onto 
the canvas while the right mouse button inserts the intial position of the truck and drone. These two types of points have been given 
different colors. Later on the points used in the background grid to discretize the domain will again be shown in a different colour. 

%{python-mode}%
\begin{flushleft} \small\label{scrap7}\raggedright\small
\NWtarget{nuweb7b}{} $\langle\,${\itshape Implementation of \verb|wrapperEnterPoints|}\nobreak\ {\footnotesize {7b}}$\,\rangle\equiv$
\vspace{-1ex}
\begin{list}{}{} \item
\mbox{}\verb@   @\\
\mbox{}\verb@def wrapperEnterPoints(fig,ax,run):@\\
\mbox{}\verb@    def _enterPoints(event):@\\
\mbox{}\verb@        if event.name      == 'button_press_event'          and \@\\
\mbox{}\verb@           run.horse_obstacle_input_mode_p  == False        and \@\\
\mbox{}\verb@           run.fly_obstacle_input_mode_p    == False        and \@\\
\mbox{}\verb@           run.common_obstacle_input_mode_p == False        and \@\\
\mbox{}\verb@           (event.button   == 1 or event.button == 3)       and \@\\
\mbox{}\verb@            event.dblclick == True and event.xdata  != None and event.ydata  != None:@\\
\mbox{}\verb@@\\
\mbox{}\verb@             if event.button == 1:  @\\
\mbox{}\verb@                 # Insert blue circle representing a site@\\
\mbox{}\verb@                 newPoint = (event.xdata, event.ydata)@\\
\mbox{}\verb@                 run.sites.append( newPoint  )@\\
\mbox{}\verb@                 patchSize  = (utils_graphics.xlim[1]-utils_graphics.xlim[0])/140.0@\\
\mbox{}\verb@                    @\\
\mbox{}\verb@                 ax.add_patch( mpl.patches.Circle( newPoint, radius = patchSize,@\\
\mbox{}\verb@                                                   facecolor='blue', edgecolor='black'  ))@\\
\mbox{}\verb@                 ax.set_title('Number of sites : ' + str(len(run.sites)), \@\\
\mbox{}\verb@                              fontdict={'fontsize':40})@\\
\mbox{}\verb@                 @\\
\mbox{}\verb@@\\
\mbox{}\verb@             elif event.button == 3:  @\\
\mbox{}\verb@                 # Insert big red circle representing initial position of horse and fly@\\
\mbox{}\verb@                 newinithorseposn  = (event.xdata, event.ydata)@\\
\mbox{}\verb@                 run.inithorseposn = newinithorseposn  @\\
\mbox{}\verb@                 patchSize         = (utils_graphics.xlim[1]-utils_graphics.xlim[0])/100.0@\\
\mbox{}\verb@@\\
\mbox{}\verb@                 ax.add_patch( mpl.patches.Circle( newinithorseposn,radius = patchSize,@\\
\mbox{}\verb@                                                   facecolor= '#D13131', edgecolor='black' ))@\\
\mbox{}\verb@                 @\\
\mbox{}\verb@                 print Fore.RED, "Initial positions of truck\n", @\\
\mbox{}\verb@                 print run.inithorseposn@\\
\mbox{}\verb@                 print Style.RESET_ALL@\\
\mbox{}\verb@@\\
\mbox{}\verb@             # Clear polygon patches and set up last minute \verb|ax| tweaks@\\
\mbox{}\verb@             clearAxPolygonPatches(ax)@\\
\mbox{}\verb@             applyAxCorrection(ax)@\\
\mbox{}\verb@             fig.canvas.draw()@\\
\mbox{}\verb@@\\
\mbox{}\verb@    return _enterPoints@\\
\mbox{}\verb@@{\NWsep}
\end{list}
\vspace{-1.5ex}
\footnotesize
\begin{list}{}{\setlength{\itemsep}{-\parsep}\setlength{\itemindent}{-\leftmargin}}
\item \NWtxtMacroRefIn\ \NWlink{nuweb3}{3}.

\item{}
\end{list}
\vspace{4ex}
\end{flushleft}
%{/python-mode}%

This function allows allows me to move a possible obstacle over the canvas before double-clicking it into place. It is called a hovering obstacle because
the position of the rectangular obstacle follows the position of the mouse cursor as it moves over the matplotlib canvas. There are three obstacle modes
one for truck-only obstacles (indicated by crimson) one for drone-only obstacles (indicated by blue) and one for obstacles common to both the truck and 
drone (indicated by green). To exit any of these obstacle placement modes to insert more sites, just press the corresponding obstacle mode keys again 
('h', 'f' and 'g') to toggle them. 

%{python-mode}%
\begin{flushleft} \small\label{scrap8}\raggedright\small
\NWtarget{nuweb8}{} $\langle\,${\itshape Implementation of \verb|wrapperHoverObstacle|}\nobreak\ {\footnotesize {8}}$\,\rangle\equiv$
\vspace{-1ex}
\begin{list}{}{} \item
\mbox{}\verb@   @\\
\mbox{}\verb@def wrapperHoverObstacle(fig,ax, run):@\\
\mbox{}\verb@        """ Wrapper for the call-back function _hoverObstacle@\\
\mbox{}\verb@        """@\\
\mbox{}\verb@        def _hoverObstacle(event, previous_patch = []):@\\
\mbox{}\verb@          """ Bind the motion of the mouse with the movement of a disk to be placed.@\\
\mbox{}\verb@          """@\\
\mbox{}\verb@          if previous_patch != []:@\\
\mbox{}\verb@              previous_patch.pop().remove() # This physically removes the patch from the screen and from memory@\\
\mbox{}\verb@@\\
\mbox{}\verb@          if event.xdata != None and event.ydata!=None and (run.horse_obstacle_input_mode_p  or @\\
\mbox{}\verb@                                                            run.fly_obstacle_input_mode_p    or @\\
\mbox{}\verb@                                                            run.common_obstacle_input_mode_p):@\\
\mbox{}\verb@              if run.horse_obstacle_input_mode_p: @\\
\mbox{}\verb@                    fcol = 'crimson'@\\
\mbox{}\verb@    @\\
\mbox{}\verb@              elif run.fly_obstacle_input_mode_p:@\\
\mbox{}\verb@                     fcol = 'blue'@\\
\mbox{}\verb@@\\
\mbox{}\verb@              elif run.common_obstacle_input_mode_p:@\\
\mbox{}\verb@                     fcol = 'green'@\\
\mbox{}\verb@    @\\
\mbox{}\verb@              current_patch = mpl.patches.Rectangle((event.xdata,event.ydata),             \@\\
\mbox{}\verb@                                                 width     = run.hovering_obstacle_width,  \@\\
\mbox{}\verb@                                                 height    = run.hovering_obstacle_height,  \@\\
\mbox{}\verb@                                                 facecolor = fcol,   \@\\
\mbox{}\verb@                                                 alpha     = 0.5)@\\
\mbox{}\verb@              previous_patch.append(current_patch)@\\
\mbox{}\verb@              ax.add_patch(current_patch)@\\
\mbox{}\verb@@\\
\mbox{}\verb@              fig.canvas.draw()@\\
\mbox{}\verb@        return _hoverObstacle@\\
\mbox{}\verb@@{\NWsep}
\end{list}
\vspace{-1.5ex}
\footnotesize
\begin{list}{}{\setlength{\itemsep}{-\parsep}\setlength{\itemindent}{-\leftmargin}}
\item \NWtxtMacroRefIn\ \NWlink{nuweb3}{3}.

\item{}
\end{list}
\vspace{4ex}
\end{flushleft}
%{/python-mode}%

When you decide to place an obstacle at a particular place on the canvas, just double-click it into place. 

%{python-mode}%
\begin{flushleft} \small\label{scrap9}\raggedright\small
\NWtarget{nuweb9}{} $\langle\,${\itshape Implementation of \verb|wrapperPlaceObstacle|}\nobreak\ {\footnotesize {9}}$\,\rangle\equiv$
\vspace{-1ex}
\begin{list}{}{} \item
\mbox{}\verb@   @\\
\mbox{}\verb@def wrapperPlaceObstacle(fig,ax,run):@\\
\mbox{}\verb@        def _placeObstacle(event):@\\
\mbox{}\verb@            """ Double-clicking Button 1 inserts the hovering disk@\\
\mbox{}\verb@                into the current arrangement and onto the canvas@\\
\mbox{}\verb@            """@\\
\mbox{}\verb@            if event.name     == 'button_press_event' and \@\\
\mbox{}\verb@               event.button   == 1                    and \@\\
\mbox{}\verb@               event.dblclick == True                 and \@\\
\mbox{}\verb@               event.xdata    != None                 and \@\\
\mbox{}\verb@               event.ydata    != None                 and \@\\
\mbox{}\verb@               (run.horse_obstacle_input_mode_p or run.fly_obstacle_input_mode_p or run.common_obstacle_input_mode_p) :@\\
\mbox{}\verb@@\\
\mbox{}\verb@                if run.horse_obstacle_input_mode_p: @\\
\mbox{}\verb@                    fcol = 'crimson'@\\
\mbox{}\verb@                    obstype = 'horseobs'@\\
\mbox{}\verb@    @\\
\mbox{}\verb@                elif run.fly_obstacle_input_mode_p:@\\
\mbox{}\verb@                     fcol = 'blue'@\\
\mbox{}\verb@                     obstype = 'flyobs'@\\
\mbox{}\verb@@\\
\mbox{}\verb@                elif run.common_obstacle_input_mode_p:@\\
\mbox{}\verb@                     fcol = 'green'@\\
\mbox{}\verb@                     obstype = 'commonobs' @\\
\mbox{}\verb@ @\\
\mbox{}\verb@                # Update the current disk list@\\
\mbox{}\verb@                run.obstacle_list.append( Obstacle( llcorner = [event.xdata,event.ydata], \@\\
\mbox{}\verb@                                             width = run.hovering_obstacle_width, \@\\
\mbox{}\verb@                                             height = run.hovering_obstacle_height, \@\\
\mbox{}\verb@                                             figure      = fig, \@\\
\mbox{}\verb@                                             axes_object = ax,@\\
\mbox{}\verb@                                             obstype     = obstype  ,@\\
\mbox{}\verb@                                             diskcolor   = fcol)  )@\\
\mbox{}\verb@@\\
\mbox{}\verb@                # Add representation of the disk appended to the canvas and show the updated count@\\
\mbox{}\verb@                ax.add_patch( run.obstacle_list[-1].mplPatch(fcol))@\\
\mbox{}\verb@@\\
\mbox{}\verb@                # Render the canvas@\\
\mbox{}\verb@                fig.canvas.draw()@\\
\mbox{}\verb@        return _placeObstacle@\\
\mbox{}\verb@@{\NWsep}
\end{list}
\vspace{-1.5ex}
\footnotesize
\begin{list}{}{\setlength{\itemsep}{-\parsep}\setlength{\itemindent}{-\leftmargin}}
\item \NWtxtMacroRefIn\ \NWlink{nuweb3}{3}.

\item{}
\end{list}
\vspace{4ex}
\end{flushleft}
%{/python-mode}%

Obstacle sizes can be increased by using the \verb|Shift| key and decreased using the \verb|Ctrl| key. 

%{python-mode}%
\begin{flushleft} \small\label{scrap10}\raggedright\small
\NWtarget{nuweb10}{} $\langle\,${\itshape Implementation of \verb|wrapperResizeHoveringObstacle|}\nobreak\ {\footnotesize {10}}$\,\rangle\equiv$
\vspace{-1ex}
\begin{list}{}{} \item
\mbox{}\verb@   @\\
\mbox{}\verb@def wrapperResizeHoveringObstacle(fig,ax,run):@\\
\mbox{}\verb@        """ Wrapper for the call-back function _resizeHoveringObstacle@\\
\mbox{}\verb@        """@\\
\mbox{}\verb@        def _resizeHoveringObstacle(event):@\\
\mbox{}\verb@            """ Each key-press increments or decrements by a fixed amount@\\
\mbox{}\verb@            the radius of the hovering disk. Change the frozenset global config@\\
\mbox{}\verb@            GC dictionary for changing the increment and decrement deltas @\\
\mbox{}\verb@            corresponding to each key-press@\\
\mbox{}\verb@            """@\\
\mbox{}\verb@@\\
\mbox{}\verb@            # This used to be in the arguments to _resizeHoveringObstacle@\\
\mbox{}\verb@            previous_patch=[]@\\
\mbox{}\verb@@\\
\mbox{}\verb@            # Increase hovering disk radius@\\
\mbox{}\verb@            if  event.key  == "shift":@\\
\mbox{}\verb@                run.hovering_obstacle_height += 0.05@\\
\mbox{}\verb@                run.hovering_obstacle_width  += 0.05@\\
\mbox{}\verb@@\\
\mbox{}\verb@                current_patch = mpl.patches.Rectangle((event.xdata,event.ydata),         \@\\
\mbox{}\verb@                                                   width     = run.hovering_obstacle_width,  \@\\
\mbox{}\verb@                                                   height    = run.hovering_obstacle_height, \@\\
\mbox{}\verb@                                                   facecolor = 'black', \@\\
\mbox{}\verb@                                                   alpha=0.2)@\\
\mbox{}\verb@                previous_patch.append(current_patch)@\\
\mbox{}\verb@                ax.add_patch(current_patch)@\\
\mbox{}\verb@@\\
\mbox{}\verb@                fig.canvas.draw()@\\
\mbox{}\verb@@\\
\mbox{}\verb@            elif event.key == "control" and \@\\
\mbox{}\verb@                run.hovering_obstacle_width >= 2.0 * 0.05:@\\
\mbox{}\verb@@\\
\mbox{}\verb@                run.hovering_obstacle_width  -= 0.05@\\
\mbox{}\verb@                run.hovering_obstacle_height -= 0.05@\\
\mbox{}\verb@@\\
\mbox{}\verb@                current_patch = mpl.patches.Rectangle((event.xdata,event.ydata),           \@\\
\mbox{}\verb@                                                   width     = run.hovering_obstacle_width,  \@\\
\mbox{}\verb@                                                   height    = run.hovering_obstacle_height, \@\\
\mbox{}\verb@                                                   facecolor = 'black', \@\\
\mbox{}\verb@                                                   alpha=0.2)@\\
\mbox{}\verb@                previous_patch.append(current_patch)@\\
\mbox{}\verb@                ax.add_patch(current_patch)@\\
\mbox{}\verb@@\\
\mbox{}\verb@                fig.canvas.draw()@\\
\mbox{}\verb@@\\
\mbox{}\verb@            while len(previous_patch) != 0:@\\
\mbox{}\verb@                previous_patch.pop().remove()@\\
\mbox{}\verb@@\\
\mbox{}\verb@        return _resizeHoveringObstacle@\\
\mbox{}\verb@@\\
\mbox{}\verb@@{\NWsep}
\end{list}
\vspace{-1.5ex}
\footnotesize
\begin{list}{}{\setlength{\itemsep}{-\parsep}\setlength{\itemindent}{-\leftmargin}}
\item \NWtxtMacroRefIn\ \NWlink{nuweb3}{3}.

\item{}
\end{list}
\vspace{4ex}
\end{flushleft}
%{/python-mode}%

\section{Discretization of the Geometric Domain}

Having placed all the obstacles in memory, we now discretize the underlying geometric domain. 
I will be using CGAL for performing Segment-Segment intersection tests while building the graphs. 

\verb|background_grid_pts| is a list of the points used in the background grid
(typically uniform grid points, or points distributed in the surrounding square
according to some probability distribution.). The input graph is created by first 
making a Delaunay Triangulation on the set of sites and \verb|background_grid_pts|. 
Then within each triangle create $k$ points on each edge where and the complete graph 
between all points on that triangle. 

%{python-mode}%
\begin{flushleft} \small
\begin{minipage}{\linewidth}\label{scrap11}\raggedright\small
\NWtarget{nuweb11a}{} $\langle\,${\itshape Definition of method \verb|makeHorseflyInputGraph|}\nobreak\ {\footnotesize {11a}}$\,\rangle\equiv$
\vspace{-1ex}
\begin{list}{}{} \item
\mbox{}\verb@   @\\
\mbox{}\verb@def makeHorseflyInputGraph(self, fig, ax, background_grid_pts, k=5):@\\
\mbox{}\verb@    @\hbox{$\langle\,${\itshape Create an initial list of nodes to be inserted into \verb|self.horsefly_input_graph|}\nobreak\ {\footnotesize \NWlink{nuweb11b}{11b}}$\,\rangle$}\verb@@\\
\mbox{}\verb@    @\hbox{$\langle\,${\itshape Extract data from the Delaunay Triangulation of the points corresponding to the nodes}\nobreak\ {\footnotesize \NWlink{nuweb12a}{12a}}$\,\rangle$}\verb@@\\
\mbox{}\verb@    @\hbox{$\langle\,${\itshape Add points along each edge of a simplex in the triangulation}\nobreak\ {\footnotesize \NWlink{nuweb12b}{12b}}$\,\rangle$}\verb@@\\
\mbox{}\verb@    @\hbox{$\langle\,${\itshape Iterate through every simplex and draw the complete graph on the points on the edges of simplex}\nobreak\ {\footnotesize \NWlink{nuweb13}{13}}$\,\rangle$}\verb@@\\
\mbox{}\verb@    @\hbox{$\langle\,${\itshape Make a list of possible edges that will be tested for intersection against the obstacles}\nobreak\ {\footnotesize \NWlink{nuweb14}{14}}$\,\rangle$}\verb@@\\
\mbox{}\verb@    @\hbox{$\langle\,${\itshape Construct the graph consisting of the edges that were filtered through}\nobreak\ {\footnotesize \NWlink{nuweb15a}{15a}}$\,\rangle$}\verb@@\\
\mbox{}\verb@@{\NWsep}
\end{list}
\vspace{-1.5ex}
\footnotesize
\begin{list}{}{\setlength{\itemsep}{-\parsep}\setlength{\itemindent}{-\leftmargin}}
\item \NWtxtMacroRefIn\ \NWlink{nuweb4}{4}.

\item{}
\end{list}
\end{minipage}\vspace{4ex}
\end{flushleft}
%{/python-mode}%

This chunk does as it says in the label. Along with recording a point, we also note its 
type whether it is a background node, a site or the initial position of the truck and drone. 
Each of these points will have different roles to play in the algorithms for horsefly on graphs. 

%{python-mode}%
\begin{flushleft} \small
\begin{minipage}{\linewidth}\label{scrap12}\raggedright\small
\NWtarget{nuweb11b}{} $\langle\,${\itshape Create an initial list of nodes to be inserted into \verb|self.horsefly_input_graph|}\nobreak\ {\footnotesize {11b}}$\,\rangle\equiv$
\vspace{-1ex}
\begin{list}{}{} \item
\mbox{}\verb@   @\\
\mbox{}\verb@node_list = []@\\
\mbox{}\verb@@\\
\mbox{}\verb@# Insert background_grid_pts@\\
\mbox{}\verb@for pt in background_grid_pts:@\\
\mbox{}\verb@        node_list.append( {'coordinates': pt, @\\
\mbox{}\verb@                           'point_type' : 'background'} )@\\
\mbox{}\verb@# Insert site points@\\
\mbox{}\verb@for site in run.sites:@\\
\mbox{}\verb@        node_list.append({'coordinates':site, @\\
\mbox{}\verb@                          'point_type': 'site'})@\\
\mbox{}\verb@@\\
\mbox{}\verb@# Insert inithorseposn if it has been provided@\\
\mbox{}\verb@if run.inithorseposn:@\\
\mbox{}\verb@         node_list.append({'coordinates':run.inithorseposn, @\\
\mbox{}\verb@                           'point_type':'inithorseposn'})@\\
\mbox{}\verb@@{\NWsep}
\end{list}
\vspace{-1.5ex}
\footnotesize
\begin{list}{}{\setlength{\itemsep}{-\parsep}\setlength{\itemindent}{-\leftmargin}}
\item \NWtxtMacroRefIn\ \NWlink{nuweb11a}{11a}.

\item{}
\end{list}
\end{minipage}\vspace{4ex}
\end{flushleft}
%{/python-mode}%

For some weird reason, the implementation of Delauny Triangulation in scipy does not 
provide me with an explicit list of edges in the computed answer. These have to be 
extracted by iterating through each simplex, and then discarding duplicate edges 
(because two simplices might share the same edge)

%{python-mode}%
\begin{flushleft} \small
\begin{minipage}{\linewidth}\label{scrap13}\raggedright\small
\NWtarget{nuweb12a}{} $\langle\,${\itshape Extract data from the Delaunay Triangulation of the points corresponding to the nodes}\nobreak\ {\footnotesize {12a}}$\,\rangle\equiv$
\vspace{-1ex}
\begin{list}{}{} \item
\mbox{}\verb@@\\
\mbox{}\verb@from scipy.spatial import Delaunay@\\
\mbox{}\verb@tri = Delaunay([node['coordinates'] for node in node_list])@\\
\mbox{}\verb@numtris = len(tri.simplices)@\\
\mbox{}\verb@@\\
\mbox{}\verb@print tri.simplices@\\
\mbox{}\verb@@\\
\mbox{}\verb@del_tri_edge_list = []@\\
\mbox{}\verb@for simplex, fidx in zip(tri.simplices,range(numtris)):@\\
\mbox{}\verb@        [i,j,k] = simplex@\\
\mbox{}\verb@        @\\
\mbox{}\verb@        del_tri_edge_list.append([[min(i,j), max(i,j)], fidx])@\\
\mbox{}\verb@        del_tri_edge_list.append([[min(j,k), max(j,k)], fidx])@\\
\mbox{}\verb@        del_tri_edge_list.append([[min(k,i), max(k,i)], fidx])@\\
\mbox{}\verb@@\\
\mbox{}\verb@del_tri_edge_list.sort()@\\
\mbox{}\verb@from itertools import groupby@\\
\mbox{}\verb@del_tri_edge_list = [elt[0] for elt in groupby(del_tri_edge_list)]@\\
\mbox{}\verb@ @\\
\mbox{}\verb@utils_algo.print_list(del_tri_edge_list)@\\
\mbox{}\verb@   @\\
\mbox{}\verb@@{\NWsep}
\end{list}
\vspace{-1.5ex}
\footnotesize
\begin{list}{}{\setlength{\itemsep}{-\parsep}\setlength{\itemindent}{-\leftmargin}}
\item \NWtxtMacroRefIn\ \NWlink{nuweb11a}{11a}.

\item{}
\end{list}
\end{minipage}\vspace{4ex}
\end{flushleft}
%{/python-mode}%

%{python-mode}%
\begin{flushleft} \small
\begin{minipage}{\linewidth}\label{scrap14}\raggedright\small
\NWtarget{nuweb12b}{} $\langle\,${\itshape Add points along each edge of a simplex in the triangulation}\nobreak\ {\footnotesize {12b}}$\,\rangle\equiv$
\vspace{-1ex}
\begin{list}{}{} \item
\mbox{}\verb@   @\\
\mbox{}\verb@def add_points_interior_to_segment(p,q,k):@\\
\mbox{}\verb@    p, q = map(np.asarray, [p,q])@\\
\mbox{}\verb@    return []@\\
\mbox{}\verb@    #return [p + float(i)/3.0 * (q-p) for i in [1,2]] @\\
\mbox{}\verb@ @\\
\mbox{}\verb@old_len_of_node_list = len(node_list)@\\
\mbox{}\verb@edges_processed      = {}@\\
\mbox{}\verb@face_info            = { fidx: [] for fidx in range(numtris)  }@\\
\mbox{}\verb@@\\
\mbox{}\verb@for [[i,j], fidx] in del_tri_edge_list:@\\
\mbox{}\verb@@\\
\mbox{}\verb@      if (i,j) not in edges_processed.keys():@\\
\mbox{}\verb@              new_segment_pts = add_points_interior_to_segment(node_list[i]['coordinates'], @\\
\mbox{}\verb@                                                       node_list[j]['coordinates'], k)@\\
\mbox{}\verb@              for pt in new_segment_pts:@\\
\mbox{}\verb@                  node_list.append({'coordinates': pt, 'point_type' : 'background'})@\\
\mbox{}\verb@    @\\
\mbox{}\verb@              num_new_nodes = len(new_segment_pts)@\\
\mbox{}\verb@@\\
\mbox{}\verb@              print Fore.RED, "New nodes added: ", range(old_len_of_node_list, old_len_of_node_list + num_new_nodes), Style.RESET_ALL@\\
\mbox{}\verb@@\\
\mbox{}\verb@              edges_processed[(i,j)] = range(old_len_of_node_list, @\\
\mbox{}\verb@                                             old_len_of_node_list+num_new_nodes)@\\
\mbox{}\verb@              face_info[fidx].append( ((i,j), edges_processed[(i,j)]) )  @\\
\mbox{}\verb@              old_len_of_node_list = len(node_list)@\\
\mbox{}\verb@      else: @\\
\mbox{}\verb@              face_info[fidx].append( ((i,j), edges_processed[(i,j)]) )  @\\
\mbox{}\verb@             @\\
\mbox{}\verb@@{\NWsep}
\end{list}
\vspace{-1.5ex}
\footnotesize
\begin{list}{}{\setlength{\itemsep}{-\parsep}\setlength{\itemindent}{-\leftmargin}}
\item \NWtxtMacroRefIn\ \NWlink{nuweb11a}{11a}.

\item{}
\end{list}
\end{minipage}\vspace{4ex}
\end{flushleft}
%{/python-mode}%


%{python-mode}%
\begin{flushleft} \small
\begin{minipage}{\linewidth}\label{scrap15}\raggedright\small
\NWtarget{nuweb13}{} $\langle\,${\itshape Iterate through every simplex and draw the complete graph on the points on the edges of simplex}\nobreak\ {\footnotesize {13}}$\,\rangle\equiv$
\vspace{-1ex}
\begin{list}{}{} \item
\mbox{}\verb@   @\\
\mbox{}\verb@    @\\
\mbox{}\verb@possible_edge_list_idxs = []@\\
\mbox{}\verb@for fidx, finfo in face_info.items(): @\\
\mbox{}\verb@@\\
\mbox{}\verb@    # Create graph edges along each simplex edge@\\
\mbox{}\verb@    # and add to possible_edge_list_idxs@\\
\mbox{}\verb@    for ((i,j), new_pts_idxs) in finfo:@\\
\mbox{}\verb@         tmp = [i] + new_pts_idxs + [j]@\\
\mbox{}\verb@         possible_edge_list_idxs.extend(zip(tmp,tmp[1:]))@\\
\mbox{}\verb@@\\
\mbox{}\verb@    # Create graph edges interior to each face between @\\
\mbox{}\verb@    # every pair of edges and add to possible_edge_list_idxs@\\
\mbox{}\verb@    [ (_, idxs1 ), (_, idxs2), (_, idxs3) ]= finfo@\\
\mbox{}\verb@    @\\
\mbox{}\verb@    print Fore.YELLOW, finfo, Style.RESET_ALL@\\
\mbox{}\verb@@\\
\mbox{}\verb@    possible_edge_list_idxs.extend([ (i,j) for i in idxs1 for j in idxs2 ])@\\
\mbox{}\verb@    possible_edge_list_idxs.extend([ (i,j) for i in idxs2 for j in idxs3 ])@\\
\mbox{}\verb@    possible_edge_list_idxs.extend([ (i,j) for i in idxs3 for j in idxs1 ])@\\
\mbox{}\verb@@\\
\mbox{}\verb@@\\
\mbox{}\verb@# Extract the actual nodes from node_list correponding to the @\\
\mbox{}\verb@# the indexes in possible_edge_list_idxs@\\
\mbox{}\verb@possible_edge_list = map(lambda xs : [node_list[xs[0]], node_list[xs[1]]], @\\
\mbox{}\verb@                         possible_edge_list_idxs)@\\
\mbox{}\verb@@{\NWsep}
\end{list}
\vspace{-1.5ex}
\footnotesize
\begin{list}{}{\setlength{\itemsep}{-\parsep}\setlength{\itemindent}{-\leftmargin}}
\item \NWtxtMacroRefIn\ \NWlink{nuweb11a}{11a}.

\item{}
\end{list}
\end{minipage}\vspace{4ex}
\end{flushleft}
%{/python-mode}%



%{python-mode}%
\begin{flushleft} \small
\begin{minipage}{\linewidth}\label{scrap16}\raggedright\small
\NWtarget{nuweb14}{} $\langle\,${\itshape Make a list of possible edges that will be tested for intersection against the obstacles}\nobreak\ {\footnotesize {14}}$\,\rangle\equiv$
\vspace{-1ex}
\begin{list}{}{} \item
\mbox{}\verb@   @\\
\mbox{}\verb@@\\
\mbox{}\verb@for node,i in zip(node_list, range(len(node_list))):@\\
\mbox{}\verb@    node['idx'] = i@\\
\mbox{}\verb@@\\
\mbox{}\verb@for pt_edge in possible_edge_list:@\\
\mbox{}\verb@@\\
\mbox{}\verb@             [p,q] = pt_edge@\\
\mbox{}\verb@             pcd = p['coordinates']@\\
\mbox{}\verb@             qcd = q['coordinates']@\\
\mbox{}\verb@@\\
\mbox{}\verb@             obstacle_interscn_info = []@\\
\mbox{}\verb@             for obs in run.obstacle_list:@\\
\mbox{}\verb@                   (interscn_p, _) = obs.intersectionWithSegment(pcd,qcd)@\\
\mbox{}\verb@                   if interscn_p:@\\
\mbox{}\verb@                       obstacle_interscn_info.append(obs.obstype)    @\\
\mbox{}\verb@ @\\
\mbox{}\verb@             # Make a count of the type of each obstacle intersected by segment@\\
\mbox{}\verb@             obs_interscn_count = {'horseobs': 0, 'flyobs':0, 'commonobs':0}@\\
\mbox{}\verb@             for obstype in obstacle_interscn_info:@\\
\mbox{}\verb@                if   obstype == 'horseobs':@\\
\mbox{}\verb@                         obs_interscn_count['horseobs']  += 1@\\
\mbox{}\verb@                 @\\
\mbox{}\verb@                elif obstype == 'flyobs':@\\
\mbox{}\verb@                         obs_interscn_count['flyobs']    += 1@\\
\mbox{}\verb@@\\
\mbox{}\verb@                elif obstype == 'commonobs':@\\
\mbox{}\verb@                         obs_interscn_count['commonobs'] += 1@\\
\mbox{}\verb@@\\
\mbox{}\verb@@\\
\mbox{}\verb@    @\\
\mbox{}\verb@             if   obs_interscn_count['horseobs']   == 0  and \@\\
\mbox{}\verb@                obs_interscn_count['flyobs']     == 0  and \@\\
\mbox{}\verb@                obs_interscn_count['commonobs']  == 0:@\\
\mbox{}\verb@@\\
\mbox{}\verb@                run.horsefly_input_graph.add_edge(p['idx'], q['idx'], edgetype='commonedge' )@\\
\mbox{}\verb@                ax.plot( [pcd[0],qcd[0]], [pcd[1],qcd[1]], 'y-',  alpha=0.2 ) @\\
\mbox{}\verb@@\\
\mbox{}\verb@@\\
\mbox{}\verb@             elif obs_interscn_count['horseobs']  >= 1  and \@\\
\mbox{}\verb@                obs_interscn_count['flyobs']      == 0  and \@\\
\mbox{}\verb@                obs_interscn_count['commonobs']   == 0:@\\
\mbox{}\verb@@\\
\mbox{}\verb@                run.horsefly_input_graph.add_edge(p['idx'], q['idx'], edgetype='flyedge' )@\\
\mbox{}\verb@                ax.plot( [pcd[0],qcd[0]], [pcd[1],qcd[1]], 'b-',  alpha=0.2 ) @\\
\mbox{}\verb@@\\
\mbox{}\verb@    @\\
\mbox{}\verb@             elif obs_interscn_count['horseobs']    == 0  and \@\\
\mbox{}\verb@                 obs_interscn_count['flyobs']      >= 1  and \@\\
\mbox{}\verb@                 obs_interscn_count['commonobs']   == 0:@\\
\mbox{}\verb@@\\
\mbox{}\verb@                 run.horsefly_input_graph.add_edge(p['idx'], q['idx'], edgetype='horseedge' )@\\
\mbox{}\verb@                 ax.plot( [pcd[0],qcd[0]], [pcd[1],qcd[1]], 'r-', alpha = 0.2 ) @\\
\mbox{}\verb@@\\
\mbox{}\verb@             else:@\\
\mbox{}\verb@                 continue @\\
\mbox{}\verb@@\\
\mbox{}\verb@@{\NWsep}
\end{list}
\vspace{-1.5ex}
\footnotesize
\begin{list}{}{\setlength{\itemsep}{-\parsep}\setlength{\itemindent}{-\leftmargin}}
\item \NWtxtMacroRefIn\ \NWlink{nuweb11a}{11a}.

\item{}
\end{list}
\end{minipage}\vspace{4ex}
\end{flushleft}
%{/python-mode}%

%{python-mode}%
\begin{flushleft} \small
\begin{minipage}{\linewidth}\label{scrap17}\raggedright\small
\NWtarget{nuweb15a}{} $\langle\,${\itshape Construct the graph consisting of the edges that were filtered through}\nobreak\ {\footnotesize {15a}}$\,\rangle\equiv$
\vspace{-1ex}
\begin{list}{}{} \item
\mbox{}\verb@   @\\
\mbox{}\verb@@{\NWsep}
\end{list}
\vspace{-1.5ex}
\footnotesize
\begin{list}{}{\setlength{\itemsep}{-\parsep}\setlength{\itemindent}{-\leftmargin}}
\item \NWtxtMacroRefIn\ \NWlink{nuweb11a}{11a}.

\item{}
\end{list}
\end{minipage}\vspace{4ex}
\end{flushleft}
%{/python-mode}%


\section{Rendering the Discretized Graph}

%{python-mode}%
\begin{flushleft} \small
\begin{minipage}{\linewidth}\label{scrap18}\raggedright\small
\NWtarget{nuweb15b}{} $\langle\,${\itshape Definition of method \verb|renderHorseflyInputGraph|}\nobreak\ {\footnotesize {15b}}$\,\rangle\equiv$
\vspace{-1ex}
\begin{list}{}{} \item
\mbox{}\verb@   @\\
\mbox{}\verb@def renderHorseflyInputGraph(self,fig,ax):@\\
\mbox{}\verb@    pass@\\
\mbox{}\verb@@{\NWsep}
\end{list}
\vspace{-1.5ex}
\footnotesize
\begin{list}{}{\setlength{\itemsep}{-\parsep}\setlength{\itemindent}{-\leftmargin}}
\item \NWtxtMacroRefIn\ \NWlink{nuweb4}{4}.

\item{}
\end{list}
\end{minipage}\vspace{4ex}
\end{flushleft}
%{/python-mode}%



\end{document}