\documentclass[10pt,english, notitlepage, oneside]{report}
\usepackage[a4paper, 
            left=1.5cm, 
            right=1.5cm,
            bottom=1.5cm,
            ]{geometry}

%-------------------------------------------------------------------------------
\usepackage[left,pagewise]{lineno}
\linenumbers
\usepackage[framemethod=TikZ]{mdframed}
\usepackage{wrapfig}
\usepackage{amsthm}
\usepackage{amsfonts}

% ------------------------------------------------------------------------------
\usepackage[toc,page]{appendix}

%-------------------------------------------------------------------------------
%https://tex.stackexchange.com/a/278199/17858
% For aligning itemize environments to left
\usepackage{enumitem}
\mdfsetup{%
   middlelinecolor=red,
   middlelinewidth=0pt,
   roundcorner=10pt}

\mdfdefinestyle{MyFrame}{%
    linecolor=black,
    outerlinewidth=0.1pt,
    %roundcorner=20pt,
    innertopmargin=14pt,
    innerbottommargin=4pt,
    innerrightmargin=4pt,
    innerleftmargin=4pt,
        leftmargin = 4pt,
        rightmargin = 4pt,
    %backgroundcolor=gray!50!white}
     }
     
%-------------------------------------------------------------------------------
\definecolor{theocol}{RGB}{255, 222, 228}
\newtheorem{theo}{Theorem}
\newenvironment{ftheo}
  {\begin{mdframed}[style=MyFrame,nobreak=true, backgroundcolor=theocol]\begin{theo}}
  {\end{theo}\end{mdframed}}


\definecolor{lemcol}{RGB}{226, 255, 220}
\newtheorem{lem}{Lemma}
\newenvironment{flem}
  {\begin{mdframed}[style=MyFrame,nobreak=true, backgroundcolor=lemcol]\begin{lem}}
  {\end{lem}\end{mdframed}}

\definecolor{corcol}{RGB}{227, 227, 227}
\newtheorem{cor}{Corollary}
\newenvironment{fcor}
  {\begin{mdframed}[style=MyFrame,nobreak=true, backgroundcolor=corcol]\begin{cor}}
  {\end{cor}\end{mdframed}}

\definecolor{propcol}{RGB}{227, 227, 227}
\newtheorem{prop}{Proposition}
\newenvironment{fprop}
  {\begin{mdframed}[style=MyFrame,nobreak=true, backgroundcolor=propcol]\begin{prop}}
  {\end{prop}\end{mdframed}}

%-------------------------------------------------------------------------------
%https://tex.stackexchange.com/a/141662/17858
%reset numbering of theorems, lemmas and corollaries every chapter 
\usepackage{chngcntr}
\counterwithin*{theo}{chapter} 
\counterwithin*{lem}{chapter} 
\counterwithin*{cor}{chapter} 
\usepackage{enumitem}

%-------------------------------------------------------------------------------
% https://tex.stackexchange.com/q/2291/17858
% if you want to create a new list from scratch
\newlist{alphalist}{enumerate}{1}
% in that case, at least label must be specified using \setlist
\setlist[alphalist,1]{label=\textbf{\Alph*.}}

%--------------------------------------------------------------------------------
%\input{../OrgWeb/litproofs.tex}
\usepackage[english]{babel}
%\setlength{\voffset}{-0.75in}
\setlength{\headsep}{5pt}

%-------------------------------------------------------------------------------
\usepackage{blindtext}
\usepackage{lipsum}
\usepackage{textcomp}

%-------------------------------------------------------------------------------
\usepackage{asymptote}
\usepackage{parskip} 
\usepackage{marginnote}
\usepackage{tocloft}
\usepackage{graphicx}                             % Include images into pdf document 
\usepackage{tikz}                                 % Useful for the circled command. Don't remove!
\usepackage{multicol}                             % For lists in two or more columns
\usepackage[ruled, algochapter]{algorithm2e}                          % Typesetting pseudo-code
\newcommand\mycommfont[1]{\footnotesize\ttfamily\textcolor{blue}{#1}}
\SetCommentSty{mycommfont}
\SetKwComment{Comment}{$\triangleright$\ }{}

%-------------------------------------------------------------------------------
\usepackage[citecolor=red]{hyperref}  % Hyper-links in latex
\usepackage{needspace}                % So that sections/code-blocks don't straddle two pages 
\usepackage{mathtools}
\usepackage{subfig}
\usepackage{etoolbox}
\usepackage{color}
\usepackage{pifont}
\setlength{\parindent}{20pt}
% For italicizing quotes: https://tex.stackexchange.com/a/288556/17858

%-------------------------------------------------------------------------------
\usepackage[utf8]{inputenc}
\usepackage{csquotes}
\renewcommand{\mkbegdispquote}[2]{\itshape}

%-------------------------------------------------------------------------------
% Convenience commands
\newcommand*\circled[1]{\tikz[baseline=(char.base)]{\node[shape=circle,draw,inner sep=2pt] (char) {#1};}}
\providecommand{\myceil}[1]{\left \lceil #1 \right \rceil }	% Ceil function
\providecommand{\myfloor}[1]{\left \lfloor #1 \right \rfloor }	% Floor function\renewcommand{\labelitemi}{\tiny$\blacksquare$}	
\newcommand\given[1][]{\:#1\vert\:}                             % for drawing the conditional probability `|` sign neatly.
\newcommand\RR{\mathbb{R}}					% Set of reals numbers
\newcommand\CC{\mathbb{C}}					% Set of complex numbers
\newcommand\ZZ{\mathbb{Z}}					% Set of integers
\newcommand\NN{\mathbb{N}}					% Set of naturals
\newcommand\rarr{\rightarrow}					% Rightarrow
\newcommand\larr{\leftarrow}					% Leftarrow
\newcommand\defeq{\coloneqq}					% := symbol
\renewcommand\tilde{ \: \thicksim \: }				% Sane tildas
\newmdenv[topline=false, bottomline=false, skipabove=\topsep,skipbelow=\topsep]{siderules}
%\setlength{\droptitle}{-18em}					% Eliminate the default vertical space
\hypersetup{colorlinks=true,linkcolor=blue}


%-------------------------------------------------------------------------------
% Writing the text of a section immediately after section title
% Vedic and old british style where text immediately follows the 
% subsection number. https://tex.stackexchange.com/a/99177/17858
\usepackage{titlesec}
\titleformat{\section}[runin]{\normalfont\Large\bfseries}{\thesection}{0.5em}{}
\titleformat{\chapter}[hang] 
{\normalfont\huge\bfseries}{\chaptertitlename\ \thechapter:}{1em}{} 
%% Custom Literate Proof Commands:
%\newcommand{\newchunk}[2]{\section{#2}\label{sec:#1}\quad}
\newcommand{\newchunk}{\section{}\quad}
%\newcommand{\newchunk}[1]{\section{}\label{sec:#1}\quad}
\newcommand{\refchunk}[1]{\ref{sec:#1}}
\titlespacing{\section}{0pt}{*0}{*0}
\newcommand{\sect}[1]{\section{#1}\quad} 
%-------------------------------------------------------------------------------
% For centeriing the titile
% https://tex.stackexchange.com/a/290449/17858
\usepackage{titling}
\renewcommand\maketitlehooka{\null\mbox{}\vfill}
\renewcommand\maketitlehookd{\vfill\null}

%-------------------------------------------------------------------------------
% To express an idea in a crunchy way.
\newcommand{\crunchy}[1]{\lbrack{} \large \textit{#1} \normalsize \rbrack} 
%-------------------------------------------------------------------------------
\makeatletter
\def\@makechapterhead#1{%
  %%%%\vspace*{50\p@}% %%% removed!
  {\parindent \z@ \raggedright \normalfont
    \ifnum \c@secnumdepth >\m@ne
        \huge\bfseries \@chapapp\space \thechapter
        \par\nobreak
        \vskip 20\p@
    \fi
    \interlinepenalty\@M
    \Huge \bfseries #1\par\nobreak
    \vskip 40\p@
  }}
\def\@makeschapterhead#1{%
  %%%%%\vspace*{50\p@}% %%% removed!
  {\parindent \z@ \raggedright
    \normalfont
    \interlinepenalty\@M
    \Huge \bfseries  #1\par\nobreak
    \vskip 40\p@
  }}
\makeatother


%\renewcommand\cftchapafterpnum{\vskip10pt}
%\renewcommand\cftsecafterpnum{\vskip15pt}

%\usepackage{tocloft}
%\renewcommand{\cftchapleader}{\cftdotfill{\cftdotsep}}

% Fancy lettering at the beginning of a chapter
% Use the command as \lettrine{T}{he} the latter gets
% typeset in capitals.
% https://tex.stackexchange.com/a/256580/17858
\usepackage{lettrine}
\usepackage{GoudyIn}

\renewcommand{\LettrineFontHook}{\GoudyInfamily{}}
\LettrineTextFont{\itshape}
\setcounter{DefaultLines}{3}%

\usepackage{fancyhdr}
\usepackage{datetime}

\fancyhf{}
\fancyfoot[L]{\today\ \currenttime}
\pagestyle{fancy}
%------------------------------------------------------------------------------- 
%% Use this when you want table of contents in two columns, very useful! 
% \usepackage[toc]{multitoc}
% \renewcommand*{\multicolumntoc}{2}
% \usepackage[noautomatic,splitindex]{imakeidx}% For making indices: https://www.sharelatex.com/learn/Indices
% % Have chapter-wise indices and a full merged index.
% % You need to give a name for each chapter though. 
% % Whenver you insert anything, you should use the modified
% % Index command as described below. 
% %% To use this see: https://tex.stackexchange.com/a/434423/17858
% \makeindex[columns=3]
% \makeindex[title=Chapter one index, name=index1]% chapter 1 index
% \makeindex[title=Chapter two index, name=index2]% chapter 2 index
% \makeindex[title=Global Index, name=fullindex]    % main index
% % \Index{item}: add "item" to chapter and full index
% \newcommand\Index[1]{%
%    \index[index\arabic{chapter}]{#1}%  add to chapter index
%    \index[fullindex]{#1}%              add to full index
%  }

%    ------------------------------------------------------------------------------

\newcommand\lam{\lambda}
\newcommand\PP{\mathcal{P}}
\newcommand\Slam{\mathcal{S}_{\lambda}}
\newcommand\QQ{\mathcal{Q}}
\newcommand\Qlam{\QQ_{\lambda}}
\newcommand\gamSlam{{\gamma}_{\Slam}}
\newcommand\gamlam{{\gamma}_{\lambda}}
\newcommand\RRnonneg{\RR_{\geq 0}}  % Non-negative reals



% -------------------------------------------------------
% https://tex.stackexchange.com/a/30450/17858
% For styling the first pages of parts and chapters of a
% book. 
\usepackage{titlesec}

\titleclass{\part}{top} % make part like a chapter
\titleformat{\part}
[display]
{\centering\normalfont\Huge\bfseries}
{\titlerule[5pt]\vspace{3pt}\titlerule[2pt]\vspace{3pt}\MakeUppercase{\partname} \thepart}
{0pt}
{\titlerule[2pt]\vspace{1pc}\huge\MakeUppercase}
%
\titlespacing*{\part}{0pt}{0pt}{20pt}
%
\titleclass{\chapter}{straight} % make chapter like a section (no newpage)
\titleformat{\chapter}
[display]
{\centering\normalfont\Huge\bfseries}
{\titlerule[5pt]\vspace{3pt}\titlerule[2pt]\vspace{3pt}\MakeUppercase{\chaptertitlename} \thechapter}
{0pt}
{\titlerule[2pt]\vspace{6pt}\huge\MakeUppercase}

\titlespacing*{\chapter}{0pt}{0pt}{40pt}

% https://tex.stackexchange.com/a/458876/17858 rounded pink rectangles around an inline word
\newcommand{\sticker}[1]{\tikz[baseline=(X.base)]\node [draw=red,fill=pink!60,semithick,rectangle,inner sep=2pt, rounded corners=3pt] (X) { {\footnotesize \color{red} #1}};}

\usepackage{mdframed}
\usepackage{xcolor}
\usepackage{soul}
\newcommand{\red}[1]{{\color{red} #1}}
\newcommand{\blue}[1]{{\color{blue} #1}}
\definecolor{OliveGreen}{rgb}{0,0.6,0}
\newcommand{\green}[1]{{\color{OliveGreen} #1}}
% The sorting=none option comes from wanting to list deliveries in order of appearance
% in the main paper. https://tex.stackexchange.com/a/337117
\usepackage[backend=bibtex, style=numeric, sorting=none]{biblatex} 
\setlength\bibitemsep{\baselineskip}
\addbibresource{/home/gaurish/Dropbox/MyWiki/research_projects/Study/Synopses/References.bib}
%--------------------------------------------------------------------------------
\title{Brief Notes on Research Papers and Proof Techniques}
\author{Gaurish Telang}
\date{}
%-------------------------------------------------------------------------------
\begin{document}
\setcounter{tocdepth}{0}
%\maketitle
%\newpage

\begin{titlepage}
   \vspace*{\stretch{1.0}}
   \begin{center}
     \textbf{ {\Huge Brief Notes on Research Papers and Proof Techniques} }\\
     \vspace{5mm}
      \large\textsl{Gaurish Telang}
   \end{center}
   \vspace*{\stretch{40.0}}
\end{titlepage}

\tableofcontents
\part{Research Papers}
\part{Topics in Combinatorial Optimization}
\part{Topics in Graph Theory}

\part{Topics in Computational Topology}
\newpage
\chapter{Basic Concepts}
\newchunk Point-set topology was invented to put analytical concepts on a stronger foundation. The motivation for that branch
has little to do with the rubber-sheet nature of the subject.
\newchunk Combinatorial Topology basically is counting at its heart. The Euler theorem is the most famous example of this
phenomenon. 
\newchunk Rubber-sheeting is precisely codified as continuous transformations in mathemtics. Topology began with the idea of
continuity. Here we study properties which are invariant under continuous transformations. Area, Perimeter are non-examples.
Examples include, the number of holes, number of boundary curves, etc. An even better meta-example is the principle of conservation
of energy. 

\newchunk The most fundamental concept in combinatorial topology is that of a \textbf{cell} which is any figure topologically equivalent
to a disk. Just pull the cheeks of the cell anywhich way, and you get a cell. 

\newchunk The first principle of combinatorial topology is to study complicated figures that can be built in some way
from simple figures. This means the soul of the subject is algorithmic? How to build a complicated figure/ or decompose
a figure into simple Lego-block like pieces. Even better think matrix factorizations or number theory factorization. 

\newchunk We restrict ourselves to figures that can be constructed from cells by gluing and pasting them along their edges. 
All but the most pathological surfaces are complexes. Typically computer scientists will restrict themselves to building
figures/surfaces which are complexes. A good way to visualizing the cutting and pasting is with the stitches along the baseball
all those stitches help in visualization. 

\newchunk Because the soul of topology is counting, you have to come up with new things of what to count.Anything that affords counting,
go for it. The most basic things are vertices, edges and faces. More precisely a cell is called a polygon when there are distinguishing
vertices along the boundary called the vertices, and there are only a finite number of them. Sections of the boundary in between
vertices are called edges. 

\newchunk A polyhedron is a complex that is topologically equivalent to a sphere. 

\newchunk Continuous transformations, are intuitively introduced as those which do not involve ripping or tearing. Put more
positively, it means transformations must preserve nearness of points. that is if a point is near a certain set, then the
transformed point must be near the transformed set. As a personal note, note that the definition never says the transformtion
should preserve the farness of points. This is a subtle point to be remembered. Preservation of nearness is all that
a continuous transformations ask for. And this preservation also leads to another invariant map on its own. i.e nearness
is the most fundamental invariant.

\newchunk \Large Nearness \normalsize : Let $P$ be a point in the plane. A neighborhood of $P$ is any circular disk (without the boundary-circle) that
contains $P$. Let $A$ be a subset of the plane. The point $P$ is called \textbf{near} the set $A$ if every neighborhood of $P$
contains a point of $A$. If $P$ is near $A$, we write $P\leftarrow A$. (point is on the left). this follows the definition of F.Riesz
a great analyst.

\newchunk \Large Continuity \normalsize : A continuous transformation from one subset $D$ of the plane to another $R$ is a function $f$ with
domain $D$ and range $R$ such that for any point $P \in D$ and set $A \subseteq D$ if $P$ is near $A$ then $f(P)$ is near
the set $f(A)$. In symbols, if $P \leftarrow A$ then $f(P) \leftarrow f(A)$.  

\newchunk Folding is a continuous transformation! Folding always involves rotating so that folding line co-incides with the $X$ or $Y$
axes, and then applying the absolute value function, and then rotating back so the folding line is back to its original position.
I like the terminology of folding over reflection. 

\newchunk Many maps are constructed Teichmuller like by a sequence of composition of maps. Think back to functional programming, in general
and parsing in particular, where you construct large maps by sticking together several little ones. 

\newchunk The following is an important construction component that can come in useful during map constructions or even when using
Diagrams say. This is called the infinite stretch and I already have another example of the infinite stretch constructed from the
$\tan(x)$ function. This allows you to map the open unit-disk to the entire plane $\RR$.
The function is $f(x,y) = \tan (\frac{\pi}{2}\sqrt{x^2+y^2}) (x,y)$. Other functions are similar, in that they involve a radial scaling
outwards. $g(x,y) = \frac{1}{1-\sqrt{x^2+y^2}} (x,y)$ is another fine one!

\newchunk With the notion of continuous transformation pinned down, we come to another fundamental concept that of a topological
transformation which allows us to define the subject more precisely. A top ological transformation is a continuous transformation
that has a continuous inverse transformation. In other words, a topological transformation is a continuous transofmration that can be
\textbf{continuously reversed or undone}. \footnote{Note the highlights!} Thus you are allowed to execute a \texttt{Ctrl+Z} instruction
when the inverse can  be done continuously. 

\newchunk Topology is the study of those properties of figures that endure when the figures are subjected to topological transformations. 
It is the study of invertible continuous transformations. A cell is thus any figure topologically equivalent to the closed disk
$D=\{P | ||P|| \leq 1\}$


\newchunk Here is a topic for a nice artificial intelligence question. The computer is given two shapes, and we need to give
an explicit continuous transformation mapping one to the other. And by explicity, I mean by some kind of formula. Assume that all
edges of the two figures are perfectly straight. 

\newchunk It happens that the proofs of many important theorems in combinatorial topology fall naturally into two parts:
a combinatorial part in which some counting argument is featured, and a point-set part in which limits and continuity is
used. For the point-set part we require some means of producing points near given sets when necessary. The concept of
compactness suits this purpose exactly because it is designed to ensure that a set has lots of near points. 

\newchunk The way connectedness is introduced in the chapter is intriguing. It perfectly encapulates what it means
for a shape like a disk or annulus to be ``of one piece''. 

\newchunk Even though there are lots and lots of exotic topologies, never forget that it is the topology of the plane
that has charted a way for the direction of the subject. That is the king topology of all, the exotic topologies just
fall out, by default but these exotic topologies are only incidental to the subject -- not essential!



\newpage

\chapter{Vector Fields}
\newchunk Where have you encountered vector fields before? Force-fields,
velocity fields, gradients of scalar fields such as pressure and height.
The latter alone are an endless source of vector fields because there are
many natural scalar functions which you can associated with each point in
the plane, (think dials! temperature is another great example). The vector
fields methods here seem to have a Eulerian flavor though rather than a
Lagrangian flavor. 

\newchunk Mathematical biology is rife with differential equations, think
stochastic or classical and many such things. Vector fields are central to
the study of \textbf{systems} of differential equations. 

\newchunk The study of vector fields coincides with the study continuous
transformations. For each continuous transformation $f: D \rarr \RR$
we can define a correponding vector field. We let $V(P)$ be the vector
from the point $P$ to the point $f(P)$. 

\newchunk Conversely a given vector field defines a continu


\newchunk There is another interesting topological property of a domain
i.e. it having the fixed point property. This is a kind of universal
definition so pay heed. A set $D$ has the fixed point-property if
\textbf{every} continuous transformation, no matter how violent, of
$D$ \textbf{into}\footnote{Mark this tiger, mark this!} $D$ has a
fixed point. This fixed point property is topological! i.e. if
$g : D \rightarrow R$ is a topological transformation(homeomorphism)
and $D$ has the fixed point property then so does $R$. 


\begin{ftheo}
  \textbf{Brouwer's fixed point theorem}: Cells have the fixed point
  property, i.e. every continuous transformation from a cell to itself
  always has atleast one fixed point. 
\end{ftheo}



\newchunk An annulus does not have the fixed point-property. i.e. you
can rotate an annulus onto itself, with no point-being fixed. 


\newchunk The Sperner's lemma is an important combinatorial property
pertaining to complexes which should motivate ideas in other fields.
It also is an excellent illustration of charging arguments and counting
a certain fixed quantity one way and then another way. 

The one-dimensional version in particular is simplest to understand
and also the one which contains the heart of the 2 dimensional
Sperner's lemma. 

\begin{ftheo}
  \textbf{Sperner's Lemma:} At least one triangle in a Sperner labeling
  receives all three labels: $A$,$B$, and $C$. 
\end{ftheo}

Note that nowhere was straightness used in the lemma, although it is
often illustrated like that. 

This lemma is not true on an triangular annulus with a triangular hole. 

\begin{ftheo}
  \textbf{Linear Transformations and Triangles} Given two triangles and
  a correpondence of the vertices from one triangle to another, there exists
  a unique affine map $y \larr Ax+b$ which maps one triangle to another according
  to the given vertex correpondence. Without the vertex correspondence there
  are 6 such transformations between the triangles in all.

  In other words, just like 5 points are sufficient to determine a conic
  curve uniquely, the action on three points is necessary and sufficient
  to determine an affine map.

  On a computer, the map can be computed in one of two ways, either by
  solving the 12x12 linear system for determining the coefficients in the
  matrix and shift vector, or by going through an intermediary of the
  fundamental triangle.

  Mapping the fundamental triangle to any arbitrary triangle is very easy. 

  Similar comments apply to higher dimensions, where you can always find
  an unique affine transformation to map one tetrahedron to another tetrahedron
  for a given vertex correspondence. 
\end{ftheo}


\newchunk As a general policy for constructing maps explicitly, when in doubt do
some rayshooting, and line-sweeping. Then use the fact that two intevals can be
mapped with ean easily determinable linear transformation. The final map is a
composition of these maps. And then you will get the formula, and a completely
well determined linear transformation and computer program. This sweeping technique is
excellent in combination with the fact that two triangles/tetrahedra can be mapped
to each other with an affine transformation. 


\newchunk Why are vector fields central to the solution of differential equations?
Because of the integral curves, they offer a map of the solution space. And the main
insight is that the critical points, the saddle points, are important structurally. 


\newchunk Orientation is also a nice thing to count. There are two orientations
always in the plane. There was another nice thing to count in Vector fields,
involving counting how many times a vector sweeps through a full orientation.
This was another counting that was done. 

\newchunk The Index theorem relates the counting of something in a higher dimension
to something in a lower dimension. The connection between something that happens in
boundary to something happening in the interior is interesting. Orientation is a specific
example of binary type. Always be on the hunts for binary or ternary types. 


\part{Topics in Approximation Algorithms}

\part{Topics in Discrete and Combinatorial
  Geometry}
\part{Topics in Computational Complexity}
\part{Notes on NP-Hardness Proofs}

\newpage
\chapter{Some Preliminary Philosophical Notes}

\newchunk 
First of all remember that the theory of NP-completeness was devised for yes-no questions. Note that 
yes-no functions look deceptively ``simple''. They are extremely powerful since many questions can be 
re-cast as yes-no questions. With that in mind burn the above picture of decision questions in mind. 

\newchunk  So all this begs the question: \emph{what kinds of decision questions do you encounter in computer science}?
Can you classify them? Here is an incomplete stab at this. Note that some of the decision questions
below can be polynomial time solvable. They also need not necessarily be in NP. Remember the technical 
definition of NP roughly says that if you answer yes to a question there has to be some sort of 
object which verifies the yesness in polynomial time. {\color{red} All the questions below, could be Common-Lisp 
predicates with their names ending in \texttt{-p}}. Useful mnemonic to remember! It is also possible that some questions
in each of the classes could be in NP and others in co-NP or whatever. It will all depend ont he specific 
questions at hand.


\begin{itemize}
\item For this class of algebraic equations does a \green{particular type of solution} exist? (3-SAT)
\item For a given input does a \green{certain structure} within it exist? (Ham-cycle)
\item For a given cost function and a number \(\mu\) is \green{$OPT \leq \mu$?} (TSP)
\item Does the given input object belong to such-and-such class? \green{recognition questions} (Is Planar?)
\item Are these two objects equal (or approximately equal) with respect to this \green{similarity measure}?
  (Trajectory similarity, graph isomorphism)
\item Does a pulleyganization exist for a given set of disks? (Did you see that non-optimization
  based geomtric questions can also be np-hard)
  \footnote{another possible meta-theorem? For many conjectured structural results which have
    not been true, it might be that they are untrue and that detecting if they have that particular
    structure is np-hard. This is clearly a FOCS'y paper :-D Even without
    them being famous conjectures, showing that "obvious" structural results can fail and
    that deciding is np-complete}
\item Is the Delaunay triangulation of a bunch of points hamiltonian?
\item The task of finding a proof for a given mathematical assertion is a search problem,
and is therefore in NP (afterall, when a formal proof of a mathematical statement) is written
out in excruciating detai, it can be checked mechanically, line by line by 
an efficient algorithm. So if P=NP there would be an efficent method to prove 
any theorem, thus eliminating the need for mathematicians.
\end{itemize}

\newchunk 

Note that you made the unconscious assumption for quite a while that geomtric optimizations
questions were the only way in which NP-completeness questions in the field. That's not true!
\footnote{possible meta-theorem? Could it be that any geometrical questions which ask whether a
  certain structure exists and which did \textbf{not} arise from a optimization question, that
  the Ham-Cycle would be involved in the hardness proof?}

\begin{center}
\includegraphics[width=15cm]{synopsesdocs/space_np.eps}
\end{center}

\begin{center}
\includegraphics[width=5cm]{synopsesdocs/np-monster.eps}
\end{center}


From now on whenever you hear of a hard-problem, make up a hideous monster like the one above.
You should picture monster from now on.Indeed once you discover that your problem is a monster,
solving it becomes more and more interested in terms of approximations fixed parameter tractable
algorithms and what not.

Next remember if you are going to make a Chuck Norris of a paper

\begin{figure}[h]
\centering
\includegraphics[width=10cm]{synopsesdocs/chuck_norris.eps}
\end{figure}


\begin{figure}[h]
\centering
\includegraphics[width=10cm]{synopsesdocs/nphardness_cartoon.eps}
\end{figure}

The above picture is the one thing you should take away from this document, if not anything else.


\begin{figure}[h]
\centering
\includegraphics[width=10cm]{synopsesdocs/circuit.eps}
\end{figure}


\section{Checklist for NP-completeness proofs}
\label{sec:org7fb4ef8}

These are the steps you need to go through when performing induction proofs. 

\begin{mdframed}
\begin{enumerate}
\item Verify that the problem you are studying is in NP.
\item Write (with pen and paper) a decision version of the question you are studying
\item Write (with pen and paper) a decision version of the question you know to be NP-complete and are reducing from.
\item Take particular note of the numerical parameters involved in the decision question, and think about any possible relation between them
The numerical parameter will have equality or inequality associated with them \(=b, \leq b, \geq b\). Take note specifically of these.
\item Make little configurations using elements of your problem. There might be wires hanging out trailing elsewhere. 
The configuration to which you will need to apply the non-existent polynomial time algorithm to your problem will 
have nice structure in the form of squares/triangles/circuits hooked up together. The final configuration will look 
\textbf{\emph{beautiful and nice.}}
\item Show that the transformation of an arbitrary instance of the NP-complete problem you are starting from to your problem 
can be accomplished in polynomial time.
\item Show that 
\begin{enumerate}
\item If the transformed instance has a solution, then the original instance of the NP-complete problem has a solution
\item If the original instance of the NP-complete problem has a solution, then the transformed instance has a solution.
\end{enumerate}
\end{enumerate}
\end{mdframed}

Go through all these 7 steps. Okay, don't forget!

\begin{itemize}
\item For the given question you should exploit peculiar (but valid) configurations of your problem and then 
somehow give a scheme to map each instance of the NP-complete problem of your choice to such peculiar configurations.
\textbf{Different configurations of parts of the input to your problem will interact peculiarly}. That's what you should
exploit to the fullest!

\item In particular, the weaving of different configurations of your input can be sold as modern art. That is true art
which exploits a rigid logical structure and is art at its most abstract! Become a painter and sell your paintings
to the museum for a million dollars. Logic becomes beautiful! If not nothing, you can use Hardness proofs to earn
a lot of money in the commercial world.

\item Thus me thinks that for starters you should start with \textbf{your} problem (\emph{not an existing NP-complete problem!}), divide 
and group valid inputs to your problem in various ways and then make those aspects of your input interact with one another. 
As you keep experimenting with peculiar configurations, keep thinking of what NP-complete problems will map to that particular 
configurations.

In fact you might want to do peculiar things like instead of taking an input \(n\), take an input of size \(kn\) for a suitable 
\(k\) and for each of the \(k\) groups assume you can solve your problem perfectly and then make the various solutions interact
in ways specific to an NP-complete problems such as for 3-SAT or some NP-complete graph theory question or planar sat or 
whatever.

\item The more the \textbf{short} NP-completeness proofs you read, the better you will get.  Note the shortness aspect of this. 
Like 20-20 cricket they will help you go through a rapid cycle or iteration of vaious ideas. Besides for a specific problem, 
you can always start mimicking an existing proof to show np-completeness until you have to come up with an innovation.
This is an old tried and tested idea that will never fail. Mimic until you are forced to innovate.

\item If you can't prove NP-completeness immediately, you will probably have to design a sequence of problems 
to make the np-completeness proof humanly possible. Thus I am guessing at first you might want to put 
even more structure on an existing NP-complete problem (which might end up being polynomial as we keep giving it more structure)
and then get the necessary gadgets and then improve from there.

Thus it helps in to know all the special subcases of those NP-complete problems which continue to remain NP-complete. 
Thus remember, in general  you don't just have an arsenal of 4-5 prolbems but many many problems which consist of 
variants on these, which keep giving more and more and more structure. That should make your NP-completeness proof much easier. 
It is better to think of your core arsenal as belonging to 4-5 classes of problems. Once you suspect that the answer might lie 
within a class, just search for particular sub0cases which are np-complete.
\end{itemize}

\begin{figure}[h]
\centering
\includegraphics[width=10cm]{synopsesdocs/classes_problems.eps}
\end{figure}

It is possible that you might have to devise a specific specialization of one of the existing classes of problems first
(which can be not that difficult) and then use \textbf{that} problem as your source problem for shewing NP-completeness.

\begin{itemize}
\item Funky pairing: That's all I am going to say.

\item The relationship between a NP-complete problem and an special case that is also NP-complete kind of reminds me of the 
relationship between Rolle's theorem and Lagrange's theorem. Here Rolle's roll is played by 3-SAT. (yes! this is nphardness now)
and that of Lagrange's by Planar 3-sat. (Note that np-completeness of planar 3-sat gives you np-completeness of 3-SAT for free)

\item SAT games of showing special cases of SAT that are np-complete are an extremely powerful game, because it deals with booleans
and interesting structures. You should become an expert on SAT from now on.

\item Planar 3-SAT should give you an idea. \textbf{Try to create a SAT game by marrying one aspect of your problem with the SATness}. Then 
you can include it into your paper and then say, oh well I did create a meta-proof that will be useful for a whole class of 
generations from now on. SAT is the most appealed to bunch for np-completeness ever! Didn't Erik Demaine create this class 
of games?

\item It might also be fun to deliberately keep devising specialized instances of a SAT problem (irrespective of whether \textbf{ultimately} that SAT-like 
problem will turn out to be polynomiall solvable.) It is a fun exercise in hardness and will lead you partially onto the correct solution.
But the outcome will be if that new \textbf{derived} instance itself is np-complete. That can be then shown by appealing to SAT and so on.
\end{itemize}

\begin{figure}[h]
\centering
\includegraphics[width=12cm]{synopsesdocs/newstrat.eps}
\end{figure}

\section{Todo list for np-completeness}
\label{sec:orgadab17e}

\begin{itemize}
\item Gary and Johnson: The main central tree
\item Planar 3-SAT reductions
\item Some of Erik Demaine's papers on games-hardness (Mario, \ldots{}).
\end{itemize}

Make these the chords of learning. You will not get the big picture immediately. But 
hammer the proofs into your head so you know each and every aspect of the proofs. They
will give you a reference to start working with. This should be sufficient to get you started
for now. Remember NP-completeness proofs are a bit like learning indefinite integration.
You need to be comortable with all the various tricks.
\section{The Importance of Cuts}
\label{sec:orgde61d5a}
A cut is a set of edges whose removal leaves a graph disconnected. It is often 
of interest to find a set of small cuts. 

\section{A Catalog of NP-complete Problems}
\label{sec:org59af7fa}

\textbf{Defn:} A search problem is NP-complete is \textbf{all} other search problems reduce to it. 

This is a strong requirement indeed: for a problem to be NP-complete it must be useful 
in solving every search problem in the world. It is remarkable that such problems exist. 
But infact they do!

A key thing I always forget is that these problems are formulated as decision problems
and thus there are budget and goals embedded in their statements. Thus when you are transforming 
one instance of a problem into another, you need to connect their budgets and goals in someway. 
Some of the problems have these budgets or goals, others dont. 

\begin{description}
\item[{Balanced Cut Problem}] Given a graph with \(n\) vertices, and a budget \(b\), partition the 
vertices into two sets \(S\) and \(T\) such that \(|S|, |T| \geq n/3\) and such that there are 
at most \(b\) edges between \(S\) and \(T\).

\item[{Integer Linear Programming}] Given a set of linear ineqalities \(Ax \leq b\) where \(A\) is 
an \(m\times n\) matric and \(b\) is an \(m\) vector find  a non-negative integer vector \(x\) 
such that \(Ax \leq b\) or report that none exists.

\item[{Clique}] Given a graph and a goal \(g\), find a set of \(g\) vertices 
such that all possible edges between them are present.

\item[{Set Cover}] Given a set \(E\) and several subsets of it, \(S_1, S_2, \ldots, S_m\) along with a 
budget \(b\). We are asked to select \(b\) of these subsets so that their
union is \(E\).

\item[{Subset Sum}] Given a parameter \(W\) and a set \(S\) of integers, is there a subset of \(S\) 
whose weights add \emph{exactly} to \(W\)? (this is a special case of the knapsack problem and is 
invaluable in many hardness proofs alongside 3-SAT)

\item[{Vertex Cover}] Given an undirected graph \(G=(V,E)\) and an integer \(k\), is there a subset 
of vertices \(S \subseteq V\) such that \(|S| \leq k\) and if \((v,w) \in E\) 
then either \(v \in S\) or \(w \in S\) or both.
\end{description}
Remember this! The complement of a vertex cover of a graph is an independent set. 
\begin{description}
\item[{Independent Set}] We are given a graph and an integer \(g\), and the aim is to find \(g\) vertices
that are independent, that is, no two of which have an edge between them.

\item[{Longest Path Problem}] Given a graph \(G\), with non-negative edge weights and two distinguished vertices
\(s\) and \(t\) along with a goal \(g\), find a simple (i.e. no vertex repeated) path from \(s\) to \(t\)
with total weight at-least \(g\). (This is also known as the taxi-cab ripoff problem.)
\item[{3-dimensional matching (3DM)}] Given a set \(T \subseteq X\times Y\times Z\) and an integer \(k\), decide
whether there exists a 3-dimensional matching \(M \subseteq T\) with \(|M|\geq k\). 
The problem is NP-complete een in the special case that \(k=|X|=|Y|=|Z|\)! This special case 
version is parameterless which might be an advantage in many situations. In the notes, of 
Vazirani and Papadimitriou, this is the version which is stated as 3dM. The former more general
case was copied from Wikipedia. Both are useful to know.

\item[{Numerical 3-dimensional matching (Numerical 3DM)}] Given by three multisets of integers 
\(X,Y,Z\), each containing \(k\)
elements, and a bound \(b\). The goal is to select a subset 
\(M\) of \(X \times Y \times Z\)  such that every integer in \(X,Y,Z\)
occurs exactly once and that for every triple \((x,y,z)\) in the subset 
\(x+y+z=b\) holds. 

(Notice that in the numerical 3dm problem, we don't have a collection of hyper-edges, ``topology'' or anything, 
 and further the sets \(X,Y,Z\) involved are all multi-sets of integers.) The basic set-up is similar but otherwise 
 the problem is different in spirit all together. The 3DM problem is combinatorial, whereas Numerical 3DM is 
 \ldots well \ldots arithmetic in nature.

\item[{3-Partition}] The problem is to decide whether a given multiset of integers 
can be partitioned into triples that all have the same sum. More precisely given a multiset of \(n=3m\)
positive integres, can \(S\) be partitioned into \(m\) triplets \(S_1,S_2, \ldots, S_m\) such that the sum of the 
numbers in each subset is equal. 
(\emph{Be careful! You are partitioning into triplets and not three sets of the same size})
\end{description}

Personally, I don't like the the term gadget. I would prefer the term ``configuration'', which is more 
descriptive and evocative of what happens during hardness proofs. Various configurations are wired together. 
For any new problem, it helps to think of little configurations immediately. I have seen Supantha doing this 
sometimes. Jump to a configuration directly when trying to prove hardness. 

Anyway, whenever you have a new swanky problem, which you have to prove hard, remember that the hardness 
will have to be shown by tiling together somewhat identical looking configurations in a neat way. There 
is always symmetry in these things. In fact, you might even construct them via Diagrams, they can indeed 
be constructed neatly. 

Further, use circuits and circles during proofs. These cycles and circuits can take many conceptual forms. 
Along with circuits, 3ness and 2ness helps a lot. You will see 3ness and 2ness occur many times in hard problems
especially thos we are reducing from. The 2ness and 3ness can be exploited in clever ways. 

It is also possible that a variable and clause gadget will be sophisticated enough that you will need 
sophisticated theorems to construct them, and clever reasoning. 

You sometimes make an unconscious mistake of thinking that an arbitrary SAT instance must always have a unique 
solution. This need not be true, although it might happen in some specific instances. In this context always 
remember \(Ax=0\) that is when \(A\) is rectangular the kernel need not be the zero-dimensional vector space. 

\begin{description}
\item[{3-SAT}] Satisfiability question but with at most 3 literals per clause. There is a restriction of 3-SAT 
	 which also happens to be NP-complete! This is under the restriction that no \emph{variable} appears 
	 in more than 3 clauses. 
Generally speaking, SAT problems have lots and lots of structures. Special kinds of clauses can result in a lot 
of beautiful structure within it.
\end{description}


\begin{description}
\item[{1-in-3 SAT}] : A collection of clauses \(C_1, \ldots, C_m\), \(m > 1\); each \(C_i\)
is a disjunction of exactly three literals. Is there a truth assignment
so that \emph{exactly one} literal is true in each \(C_i\)?
\end{description}



\begin{description}
\item[{Zero One Integer Programming (ZOE)}] The goal is to find a vector \(x\) of 0's and 1's satisfying \(Ax=\mathbf{1}\)
where \(A\) is a \(m \times n\) matrix with 0-1 entries and \(\mathbf{1}\) is the \(m\) - vector of all 1's. 
ZOE is an important special case of ILP.

\item[{Traveling Salesman Problem (TSP)}] Remember the classic TSP there is no graph, no euclidean plane, no nothing. 
Just a distance matrix between the nodes.
\end{description}

\section{Hardness proofs}
\label{sec:orgd0dfccc}
\begin{description}
\item[{3-SAT \(\leq_p\) Independent Set}] return to this proof always when in doubt. This is the first 
non-trivial hardness proof which I have grasped completely. Also it explains a lot of details.

\item[{3-SAT \(\leq_p\) 3DM}] Loveliest proof ever which brings out the need for clause gadgets and 
variable gadgets.
\end{description}




\begin{appendices}
\chapter{Philosophy}

 \lettrine{T}{he} goal is to demonstrate the intricate details of proofs for reuse in future papers
 following the ideas of literate programming and test-driven development. Every idea
 no matter how small or trivial is given in the form of a small chunk followed by a detailed
 explanation possibly with some examples. The key-switch in thinking about proofs is that they
 should be viewed as algorithms, even if inefficient and existential. We also uses the
 pseudo-code syntax as described by \texttt{algorithm2e}.

 \newchunk This will help in nice exposition of ideas and make ideas super crunchy. This will be useful
 in recycling ideas used from the past. You can remember this by continuously flipping through
 document. Further, you these code-blocks are all collected at the end in the form of an index.

  Within a chapter, corollaries, lemmas, algorithms and theorems, are named in serial order
 one after another. The numbering restarts at a new chapter. Remember a chapter is like a module
 self-contained which leads towards something else. Remember CLAT, CLAT! system of writing a book
 or an article or an idea. Every CLAT, should have a small explnataroy note before digging down
 into the meat and potatoes, possibly with some explnatory examples or figures. 

 \newchunk The same style is used in writing out the Horsefly, Maximally Dispersed Guards, Lookahead
 Only the reference.bib/citation styles will change. That's why I have created a standard\_header.tex
 file which will help in writing out proofs, which contains all the dirty gunky code needed to
 format the \LaTeX file to your specifications.
 {\footnotesize \color{red} See the \verb|compile| script I've written in in this folder }
 Later transfer this to other projects. 
 
 \newchunk Most sections are unlabelled, but when you want to start a new idea development, just
 name the section inline and continue with writing it. In the table of contents, only include
 those contents which are named. Each algorithm if expressed in stepwise form should be done
 in alphabetical steps. That way, you know that the number of steps is limited to less than 26,
 break up if it overshoots that. 

 Sections are generated with a \texttt{newchunk} command. Before stating a theorem, give a
 quick overview of what the theorem is about. Use asymptote code in LaTeX with the links to scripts
 to embed everything in it. The hope by doing all this is to isloate all small critical morsels of ideas.
 The idea is to treat a proof as a literate program. Steps in a proof should be labelled alphabetically. 

 Proofs are programs. Write proofs with forward references in \LaTeX via \verb|\ref{}|, and a short explanation
 of what you want to achieve, and then in a later section below, give all the details. Inside your notes, there
 are no appendices as such as, the function when forwarded referenced are explained as soon as possible.
 Don't forget this. 

 \newchunk All references, and I mean \textbf{(all fucking references!)} are put into References.bib in this folder
  across all problems. 
 
\newchunk Use \texttt{flem, fcor, ftheo} for coloring the boxes, and the Algorithm2e environment
for writing pseudo-code. Pseudo-code is either written in Knuthian form in an alphabetized
list with a \texttt{crunchy} command explaining in short what the step is doing, or in
algorithm2e form. These will typically be mutually exclusive. Some algorithms are better
described one way, and some others in another form. 

\newchunk

\begin{description}
\item[Fundae] One web-file to dump eveyrthing math philosphy fundae/random notes software
\item[Project]  One web-file/notebook for each research project. (Unique color for each project)
\item[Synopses] One web-file/(several notebooks same appearance)
  \begin{itemize}
  \item All notebooks corresponding to a project have the same color
  \item All notebooks must have page numbers
  \item All notebooks must have little stickers referring to the notebook number under use. 
  \end{itemize}
\end{description}

\begin{figure}
  \centering
  \includegraphics[width=15cm]{synopsesdocs/workflow.pdf}
\end{figure}

Photographs will have three-phase names. \verb|<project-name,synopses>-booknumber-pagenumber.jpg|>

Further, you should feel like writing in those books. Those spiral
bound books are ugly as hell, and you can't cover them to get rid of the 
grime and all. 
All notebooks (except for those which you already have) must be large, 
blank, 100 pages and have page numbers. The page numbers are important 
if there is a sophisticated construction, or diagram that you want to 
quickly reference, without going through all the take photo bull-shit. 
It can get extremely irritating, if possible name each photo with the 
name of the book and page-number. But in general, try to make as many 
images as possible in Inkscape, because they will be used one-day in 
papers anyway and when you need extreme precision.
	      
For each notebook and paper you download, get the bibtex reference label 
and place a little label, a short sticker if you will so everything can be 
serialized. Place the bibtex reference in the main `Reference.bib` file. 
Everytime, you get a paper, stick a label onto it. You probably should get 
all your paper print-outs bound into books, or at-least collected and put 
into a big-binder, which has its own table of contents listing all the 
papers added, just add the name. 

The notebooks themselves will be stacked alphabetically, possibly
rubber-bound and shelved for future reference when the stack grows really long. 

The \verb|\sticker{}| command draws a light pink box around piece of text. In
Emacs presssing \verb|Alt+F1| generates a sticker tag with a random alphanumeric
string inside it. The sticker function itself is defined in \verb|standard_header.tex|

\red{Blah blah blah}
\green{Blah blah blah}
\blue{Blah blah blah}
\hl{Blah blah blah}


\begin{center}
{\large The quick brown \sticker{sx3goz} jumped over the lazy dog. }
\end{center}

The sticker will typically be applied just in front of a newchunk. The sticker names
can then be pdfgrepped! \verb|-n| option to pdfgrep gives page-numbers and not
line-numbers unlike regular grep. 

Finally this is how you proceed: weave the web in your notebooks and then finally
tangle the web into the pdf document. The pdf document can then be treated as
executable code for your brain. The noteboooks are your rough-work and ideas
developed during deep-work, the computer files are used to commit. 


\begin{ftheo}
   \blindtext[1]
\end{ftheo}
 \begin{proof}
\begin{alphalist}
  \item \blindtext[1]
  \item \blindtext[1]
\end{alphalist}
\end{proof}

\newchunk
\begin{flem}
   \blindtext[1]
\end{flem}
 \begin{proof}
\begin{alphalist}
  \item \blindtext[1]
  \item \blindtext[1]
\end{alphalist}
\end{proof}

\newchunk
\begin{fcor}
   \blindtext[1]
\end{fcor}
 \begin{proof}
\begin{alphalist}
  \item \crunchy{Get The Variable Here} \blindtext[1]
  \item \crunchy{Find Toast} \blindtext[1]
  \item \crunchy{Make An Omelette} \blindtext[1]
\end{alphalist}
\end{proof}

\newchunk
\begin{fprop}
   \blindtext[1]
\end{fprop}
 \begin{proof}
\begin{alphalist}
  \item \crunchy{Get The Variable Here} \blindtext[1]
  \item \crunchy{Find Toast} \blindtext[1]
  \item \crunchy{Make An Omelette} \blindtext[1]
\end{alphalist}
\end{proof}

\newchunk Method 1 of expressing algorithms. (Preferred, when possible)

Method 2 is good when you want to be expressly very precise in what you
are doing. some complicated argument is going on, such as when expressing
some liner algebra algorithms like GMRES. 

\begin{algorithm}[h]
\caption{How to write algorithms}
\KwIn{A bunch of points}
\KwOut{The Convex Hulls}
% Algorithm body here!
\begin{alphalist}
\item \crunchy{Get The Variable Here} You can use it for short words without
  penalty, but for longer text this issues arise. Note that verbatim mode is
  fragile and \verb doesn't work inside macro arguments.
\item \crunchy{Find Toast} the other spaces in the same line in order to
  compensate for a longer verb box, but this has its limits. If you have longer
\item \crunchy{Make An Omelette}  The fact that it is printed in tt font is
  just a side-effect! If you use spaces inside verbatim text you are
  requesting a verbatim, unstretchable and unbreakable space. 
  \item \crunchy{Eat it} blah blah ablah

\end{alphalist}
\end{algorithm}

\newchunk \blindtext

\begin{algorithm}[h]
\SetAlgoLined
\KwData{this text}
\KwResult{how to write algorithm with \LaTeX2e }
initialization\;
\tcp{A comment!}
\While{not at end of this document}{
  read current\;
  \eIf{understand}{
    go to next section\;
    current section becomes this one\;
  }{
    go back to the beginning of current section\;
   blah blah blah  \Comment*[c]{Some comment} 
   blah blah blah  \Comment{Some other comment}
  }
}
\caption{How to write algorithms}
\end{algorithm}

The Yasnippet crunchy was installed in \newline
\verb|/home/gaurish108/.emacs.d/elpa/yasnippet-classic-snippets-1.0.2/snippets/latex-mode|

Note that \verb|\item| almost appears before \verb|\crunchy|. 

\begin{verbatim}
# contributor: Gaurish Telang
# name: \crunchy{...}
# key: cr
# --
\item \crunchy{$1}%   
\end{verbatim}

% \begin{figure}
%   \centering
%   \begin{asy}
%     size(350);

%     real a=3;
%     real b=4;
%     real c=hypot(a,b);

%     transform ta=shift(c,c)*rotate(-aCos(a/c))*scale(a/c)*shift(-c);
%     transform tb=shift(0,c)*rotate(aCos(b/c))*scale(b/c);

%     picture Pythagorean(int n) {
%       picture pic;
%       fill(pic,scale(c)*unitsquare,1/(n+1)*green+n/(n+1)*brown);
%       if(n == 0) return pic;
%       picture branch=Pythagorean(--n);
%       add(pic,ta*branch);
%       add(pic,tb*branch);
%       return pic;
%     }

%     add(Pythagorean(12));
    
%   \end{asy}
% \end{figure}

%%%%%%%%%%%%%%%%%%%%%%%%%%%%%%%%%%%%%%
% A histogram.
%%%%%%%%%%%%%%%%%%%%%%%%%%%%%%%%%%%%%% 
\blindtext
\begin{figure}
  \centering
  \begin{asy}
    import graph;
    import stats;

    size(250,250,IgnoreAspect);

    int n=100;
    real[] a=new real[n];
    for(int i=0; i < n; ++i) a[i]=Gaussrand();

    draw(graph(Gaussian,min(a),max(a)),blue);

    // Optionally calculate "optimal" number of bins a la Shimazaki and Shinomoto.
    int N=bins(a);

    histogram(a,min(a),max(a),N,normalize=true,low=0,blue,black,bars=false);

    xaxis("$x$",BottomTop,LeftTicks);
    yaxis("$dP/dx$",LeftRight,RightTicks(trailingzero));
  \end{asy}
  \caption{My great histogram}
\end{figure}

\blindtext
\end{appendices}

\nocite{*}
\printbibliography
\end{document}