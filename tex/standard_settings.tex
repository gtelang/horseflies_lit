\usepackage[a4paper, left=2cm, right=2cm, top=2cm]{geometry}
%\usepackage[text={16cm,24cm}]{geometry}
\usepackage{marginnote}


% https://stackoverflow.com/a/27243065/505306
% Making caption font smaller on figures and tables.
\usepackage{caption}
\captionsetup{font=footnotesize}

% Forcing line-breaks in url's
% https://tex.stackexchange.com/questions/3033/forcing-linebreaks-in-url
\usepackage[hyphens]{url}

%-------------------------------------------------------------------------------
%\usepackage[left,pagewise]{lineno}
%\linenumbers
\usepackage[framemethod=TikZ]{mdframed}
\usepackage{wrapfig}
\usepackage{amsthm}
\usepackage{amsfonts}
\usepackage{minted}
% ------------------------------------------------------------------------------
\usepackage[toc,page]{appendix}

%-------------------------------------------------------------------------------
%https://tex.stackexchange.com/a/278199/17858
% For aligning itemize environments to left
\usepackage{enumitem}
\mdfsetup{%
   middlelinecolor=red,
   middlelinewidth=0pt,
   roundcorner=10pt}

\mdfdefinestyle{MyFrame}{%
    linecolor=black,
    outerlinewidth=0.1pt,
    roundcorner=0pt,
    innertopmargin=14pt,
    innerbottommargin=4pt,
    innerrightmargin=4pt,
    innerleftmargin=4pt,
        leftmargin = 4pt,
        rightmargin = 4pt,
    %backgroundcolor=gray!50!white}
     }
     
%-------------------------------------------------------------------------------

\definecolor{theocol}{RGB}{255, 222, 228}
\newtheorem{theo}{Theorem}
\newenvironment{ftheo}
  {\begin{mdframed}[style=MyFrame,nobreak=true, backgroundcolor=theocol]\begin{theo}}
  {\end{theo}\end{mdframed}}


\definecolor{lemcol}{RGB}{226, 255, 220}
\newtheorem{lem}{Lemma}
\newenvironment{flem}
  {\begin{mdframed}[style=MyFrame,nobreak=true, backgroundcolor=lemcol]\begin{lem}}
  {\end{lem}\end{mdframed}}

\definecolor{corcol}{RGB}{227, 227, 227}
\newtheorem{cor}{Corollary}
\newenvironment{fcor}
  {\begin{mdframed}[style=MyFrame,nobreak=true, backgroundcolor=corcol]\begin{cor}}
  {\end{cor}\end{mdframed}}

\definecolor{propcol}{RGB}{227, 227, 227}
\newtheorem{prop}{Proposition}
\newenvironment{fprop}
  {\begin{mdframed}[style=MyFrame,nobreak=true, backgroundcolor=propcol]\begin{prop}}
  {\end{prop}\end{mdframed}}

\definecolor{notecol}{RGB}{255, 152, 149  }
\newenvironment{note}
  {\begin{mdframed}[style=MyFrame,nobreak=true, backgroundcolor=notecol]}
  {\end{mdframed}}

\usepackage{enumitem}

%-------------------------------------------------------------------------------
% https://tex.stackexchange.com/q/2291/17858
% if you want to create a new list from scratch
\newlist{alphalist}{enumerate}{1}
% in that case, at least label must be specified using \setlist
\setlist[alphalist,1]{label=\textbf{\Alph*.}}

%--------------------------------------------------------------------------------
\usepackage[english]{babel}
%\setlength{\voffset}{-0.75in}
\setlength{\headsep}{5pt}

%-------------------------------------------------------------------------------
\usepackage{blindtext}
\usepackage{lipsum}
\usepackage{textcomp}

%-------------------------------------------------------------------------------
\usepackage{asymptote}
\usepackage{asypictureB}
\usepackage{filecontents}
\usepackage{parskip} 
\usepackage{tocloft}
\usepackage{graphicx}                             % Include images into pdf document 
\usepackage{tikz}                                 % Useful for the circled command. Don't remove!
\usepackage{multicol}                             % For lists in two or more columns

\usepackage[nokwfunc, ruled]{algorithm2e}
\newcommand\mycommfont[1]{\footnotesize\ttfamily\textcolor{blue}{#1}}
\SetCommentSty{mycommfont}
\SetKwComment{Comment}{$\triangleright$\ }{}
\usepackage[T1]{fontenc}
%-------------------------------------------------------------------------------
\usepackage{needspace}                % So that sections/code-blocks don't straddle two pages 
\usepackage{mathtools}
\usepackage{subfig}
\usepackage{etoolbox}
\usepackage{color}
\usepackage{pifont}
\setlength{\parindent}{20pt}
% For italicizing quotes: https://tex.stackexchange.com/a/288556/17858

%-------------------------------------------------------------------------------
\usepackage[utf8]{inputenc}
\usepackage{csquotes}
\renewcommand{\mkbegdispquote}[2]{\itshape}

%-------------------------------------------------------------------------------
% Convenience commands
\newcommand*\circled[1]{\tikz[baseline=(char.base)]{\node[shape=circle,draw,inner sep=2pt] (char) {#1};}}
\providecommand{\myceil}[1]{\left \lceil #1 \right \rceil }	% Ceil function
\providecommand{\myfloor}[1]{\left \lfloor #1 \right \rfloor }	% Floor function\renewcommand{\labelitemi}{\tiny$\blacksquare$}	
\newcommand\given[1][]{\:#1\vert\:}                             % for drawing the conditional probability `|` sign neatly.
\newcommand\RR{\mathbb{R}}					% Set of reals numbers
\newcommand\CC{\mathbb{C}}					% Set of complex numbers
\newcommand\ZZ{\mathbb{Z}}					% Set of integers
\newcommand\NN{\mathbb{N}}					% Set of naturals
\newcommand\rarr{\rightarrow}					% Rightarrow
\newcommand\larr{\leftarrow}					% Leftarrow
\newcommand\defeq{\coloneqq}					% := symbol
\renewcommand\tilde{ \: \thicksim \: }				% Sane tildas
\newmdenv[topline=false, bottomline=false, skipabove=\topsep,skipbelow=\topsep]{siderules}
%\setlength{\droptitle}{-18em}					% Eliminate the default vertical space
%\hypersetup{colorlinks=true,linkcolor=blue}


%-------------------------------------------------------------------------------
% Writing the text of a section immediately after section title
% Vedic and old british style where text immediately follows the 
% subsection number. https://tex.stackexchange.com/a/99177/17858
%\usepackage{titlesec}
%\titlespacing{\section}{0pt}{*0}{*0}
%\titlespacing{\subsection}{0pt}{*0}{*0}
%\titlespacing{\subsubsection}{0pt}{*0}{*0}
\usepackage[compact]{titlesec}
\titlespacing{\section}{0pt}{2ex}{1ex}
\titlespacing{\subsection}{0pt}{1ex}{0ex}
\titlespacing{\subsubsection}{0pt}{0.5ex}{0ex}
%\setlength{\parskip}{0.1cm} % Paragraph separation
\setlength{\parindent}{2em}

\titleformat*{\section}{\Huge\bfseries}
\titleformat{\subsection}[runin]{\normalfont\Large\bfseries}{\thesubsection}{0.5em}{}
\titleformat*{\subsubsection}{\Large\bfseries}
%{\normalfont\huge\bfseries}{\sectiontitlename\ \thesection:}{1em}{} 
%% Custom Literate Proof Commands:
%\newcommand{\newchunk}[2]{\section{#2}\label{sec:#1}\quad}
\newcommand{\newchunk}{\subsection{}\quad}
\newcommand{\refchunk}[1]{\ref{sec:#1}}
%\newcommand{\sect}[1]{\section{#1}\quad} 
%-------------------------------------------------------------------------------
% For centeriing the titile
% https://tex.stackexchange.com/a/290449/17858
\usepackage{titling}
\renewcommand\maketitlehooka{\null\mbox{}\vfill}
\renewcommand\maketitlehookd{\vfill\null}

%-------------------------------------------------------------------------------
% To express an idea in a crunchy way.
\newcommand{\crunchy}[1]{\lbrack{} \large \textit{#1} \normalsize \rbrack} 
%-------------------------------------------------------------------------------
\makeatletter
\def\@makechapterhead#1{%
  %%%%\vspace*{50\p@}% %%% removed!
  {\parindent \z@ \raggedright \normalfont
    \ifnum \c@secnumdepth >\m@ne
        \huge \bfseries \@chapapp\space \thechapter
        \par\nobreak
        \vskip 20\p@
    \fi
    \interlinepenalty\@M
    \Huge \bfseries #1\par\nobreak
    \vskip 40\p@
  }}
\def\@makeschapterhead#1{%
  %%%%%\vspace*{50\p@}% %%% removed!
  {\parindent \z@ \raggedright
    \normalfont
    \interlinepenalty\@M
    \Huge \bfseries  #1\par\nobreak
    \vskip 40\p@
  }}
\makeatother


\usepackage{fancyhdr}
\usepackage{datetime}

\fancyhf{}
\fancyfoot[L]{\today\ \currenttime}
\pagestyle{fancy}

% https://tex.stackexchange.com/a/458876/17858 rounded
% pink rectangles around an inline word
\newcommand{\sticker}[1]{\tikz[baseline=(X.base)]\node [draw=red,fill=pink!60,semithick,rectangle,inner sep=2pt, rounded corners=3pt] (X) { {\footnotesize \color{red} #1}};}


\newif\ifshowcode
\showcodetrue
\usepackage{latexsym}
\usepackage{listings}
\usepackage{color}
\definecolor{linkcolor}{rgb}{0, 0, 0.7}

\usepackage[%
backref,%
raiselinks,%
pdfhighlight=/O,%
pagebackref,%
hyperfigures,%
breaklinks,%
colorlinks,%
pdfpagemode=None,%
pdfstartview=FitBH,%
linkcolor={linkcolor},%
anchorcolor={linkcolor},%
citecolor={linkcolor},%
filecolor={linkcolor},%
menucolor={linkcolor},%
pagecolor={linkcolor},%
urlcolor={linkcolor}%
]{hyperref}


% % %-------------------------------------------------------------------------------
% % %https://tex.stackexchange.com/a/141662/17858
% % %reset numbering of theorems, lemmas and corollaries every chapter 
% \usepackage{chngcntr}
% \counterwithin*{theo}{chapter} 
% \counterwithin*{lem}{chapter} 
% \counterwithin*{cor}{chapter} 


%% For making lualatex work with asymptote
%% ripped from https://tex.stackexchange.com/a/376412/17858
\usepackage{ifluatex}
\ifluatex
\makeatletter
\def\asy@input@graphic{%
  \ifASYinline
    \IfFileExists{"\AsyFile.tex"}{%
      \catcode`:=12\relax
      \@@input"\AsyFile.tex"\relax
    }{%
      \PackageWarning{asymptote}{file `\AsyFile.tex' not found}%
    }%
  \else
    \IfFileExists{"\AsyFile.\AsyExtension"}{%
      \ifASYattach
        \ifASYPDF
          \IfFileExists{"\AsyFile+0.pdf"}{%
            \setbox\ASYbox=\hbox{\includegraphics[hiresbb]{\AsyFile+0.pdf}}%
          }{%
            \setbox\ASYbox=\hbox{\includegraphics[hiresbb]{\AsyFile.pdf}}%
          }%
        \else
          \setbox\ASYbox=\hbox{\includegraphics[hiresbb]{\AsyFile.eps}}%
        \fi
        \textattachfile{\AsyFile.\AsyExtension}{\phantom{\copy\ASYbox}}%
        \vskip-\ht\ASYbox
        \indent
        \box\ASYbox
      \else
        \ifASYPDF
          \includegraphics[hiresbb]{\AsyFile.pdf}%
        \else
          \includegraphics[hiresbb]{\AsyFile.eps}%
        \fi
      \fi
    }{%
      \IfFileExists{"\AsyFile.tex"}{%
        \catcode`:=12
        \@@input"\AsyFile.tex"\relax
      }{%
        \PackageWarning{asymptote}{%
          file `\AsyFile.\AsyExtension' not found%
        }%
      }%
    }%
  \fi
}
\makeatother
\fi

%%%%%%%%%%%%%%%%%%%%%%%%%%%%%%%%%%%%%%%%%%%%%%%%%%%%%%%%%%%%%%%%
%% for setting program fonts
%% As recommended by http://nepsweb.co.uk/docs/progfonts.pdf
%% He recommends bera mono. 
%%\usepackage[scaled]{beramono}
%%\usepackage[T1]{fontenc}

%% I liked txtt also. 
\renewcommand{\ttdefault}{txtt}